\chapter{Instalación y configuración}

\section{Requisitos del sistema}

Antes de proceder con la instalación de SimAS 3.0, es importante verificar que el sistema cumple con los requisitos necesarios para el correcto funcionamiento de la aplicación.

\subsection{Requisitos mínimos}

Para ejecutar SimAS 3.0, el sistema debe cumplir los siguientes requisitos mínimos:

\begin{itemize}
    \item \textbf{Sistema operativo}: Windows 10/11, macOS 10.14+, o Linux (Ubuntu 18.04+).
    \item \textbf{Java Runtime Environment}: versión 17 o superior \cite{java}.
    \item \textbf{Memoria RAM}: 512 MB disponibles.
    \item \textbf{Espacio en disco}: 100 MB libres.
    \item \textbf{Resolución de pantalla}: 1024x768 píxeles mínimo.
\end{itemize}

\subsection{Requisitos recomendados}

Para obtener un rendimiento óptimo, se recomienda:

\begin{itemize}
    \item \textbf{Memoria RAM}: 2 GB o más.
    \item \textbf{Resolución de pantalla}: 1366x768 píxeles o superior.
    \item \textbf{Procesador}: dual-core 2.0 GHz o superior.
    \item \textbf{Espacio en disco}: 200 MB libres (para almacenar gramáticas y/o informes).
\end{itemize}

\section{Instalación de Java}

SimAS 3.0 requiere Java Runtime Environment (JRE) versión 17 o superior \cite{java}. Si no tiene Java instalado o tiene una versión anterior, siga estos pasos:

\subsection{Verificación de Java}

Para verificar si Java está instalado y qué versión tiene:

\begin{itemize}
    \item \textbf{Windows}: abra el símbolo del sistema (cmd) y ejecute: \texttt{java -version}.
    \item \textbf{macOS}: abra Terminal y ejecute: \texttt{java -version}.
    \item \textbf{Linux}: abra una terminal y ejecute: \texttt{java -version}.
\end{itemize}

\subsection{Descarga e instalación de Java}

Si no tiene Java instalado o tiene una versión anterior a la 17:

\begin{enumerate}
    \item Visite el sitio web oficial de Java: \cite{java}.
    \item Descargue Java 17 LTS (Long Term Support) para su sistema operativo.
    \item Ejecute el instalador y siga las instrucciones.
    \item Reinicie el sistema si es necesario.
    \item Verifique la instalación ejecutando \texttt{java -version} nuevamente.
\end{enumerate}

\textbf{Nota}: Como alternativa de código abierto, también puede descargar Java desde \cite{adoptium}.

\section{Descarga de SimAS 3.0}

\subsection{Ubicación de los archivos}

SimAS 3.0 se distribuye como una aplicación independiente que incluye todas las dependencias necesarias. Los archivos de distribución se encuentran en:

\begin{itemize}
    \item \textbf{Aplicación ejecutable}: \texttt{SimAS.app} (macOS) o \texttt{SimAS.exe} (Windows).
    \item \textbf{Archivo JAR}: \texttt{SimAS.jar} (para ejecución manual).
    \item \textbf{Documentación}: manual de usuario y archivos de ayuda.
\end{itemize}

\subsection{Verificación de la descarga}

Antes de proceder con la instalación, verifique que:

\begin{itemize}
    \item Los archivos se descargaron completamente.
    \item No hay errores de corrupción en los archivos.
    \item Tiene permisos de lectura y ejecución en los archivos.
\end{itemize}

\section{Instalación de SimAS 3.0}

\subsection{Instalación en macOS}

Para instalar SimAS 3.0 en macOS:

\begin{enumerate}
    \item Localice el archivo \texttt{SimAS.app} en su carpeta de descargas.
    \item Arrastre \texttt{SimAS.app} a la carpeta \texttt{Aplicaciones}.
    \item Opcionalmente, puede crear un alias en el escritorio para acceso rápido.
    \item La aplicación estará disponible en el Launchpad y en la carpeta Aplicaciones.
\end{enumerate}

\subsection{Instalación en Windows}

Para instalar SimAS 3.0 en Windows:

\begin{enumerate}
    \item Localice el archivo \texttt{SimAS.exe} en su carpeta de descargas.
    \item Haga clic derecho sobre el archivo y seleccione \string"Ejecutar como administrador\string".
    \item Siga las instrucciones del instalador.
    \item La aplicación se instalará en la carpeta \texttt{Program Files}.
    \item Se creará un acceso directo en el escritorio y en el menú Inicio.
\end{enumerate}

\subsection{Instalación en Linux}

Para instalar SimAS 3.0 en Linux:

\begin{enumerate}
    \item Abra una terminal y navegue a la carpeta donde descargó SimAS 3.0.
    \item Haga el archivo ejecutable: \texttt{chmod +x SimAS.jar}.
    \item Cree un directorio para la aplicación: \texttt{sudo mkdir /opt/simas}.
    \item Copie los archivos: \texttt{sudo cp -r * /opt/simas/}.
    \item Cree un acceso directo en el escritorio o menú de aplicaciones.
\end{enumerate}

\section{Primera ejecución}

\subsection{Inicio de la aplicación}

Para ejecutar SimAS 3.0 por primera vez:

\begin{enumerate}
    \item \textbf{macOS}: haga doble clic en \texttt{SimAS.app} o búsquelo en Launchpad.
    \item \textbf{Windows}: haga doble clic en el acceso directo del escritorio o búsquelo en el menú Inicio.
    \item \textbf{Linux}: ejecute \texttt{java -jar /opt/simas/SimAS.jar} o use el acceso directo creado.
\end{enumerate}

\subsection{Pantalla de bienvenida}

La primera vez que ejecute SimAS 3.0, verá:

\begin{itemize}
    \item La pantalla de bienvenida con el logo de la aplicación.
    \item Información sobre la versión y el desarrollador.
    \item Un indicador de progreso mientras se cargan los componentes.
\end{itemize}

\subsection{Configuración inicial}

Después de la primera ejecución:

\begin{enumerate}
    \item La aplicación se abrirá en el menú principal.
    \item Se establecerán los valores predeterminados para la interfaz.
    \item El idioma se configurará en español por defecto.
    \item La aplicación estará lista para crear y gestionar gramáticas.
\end{enumerate}

\section{Configuración de la aplicación}

\subsection{Selección de idioma}

SimAS 3.0 soporta múltiples idiomas. Para cambiar el idioma:

\begin{enumerate}
    \item Localice el selector de idioma en la esquina superior derecha.
    \item Haga clic en el menú desplegable.
    \item Seleccione el idioma deseado de la lista disponible.
    \item La interfaz se actualizará inmediatamente al nuevo idioma.
\end{enumerate}

\subsection{Gestión de archivos}

SimAS 3.0 permite guardar y cargar archivos en cualquier ubicación de su sistema:

\begin{itemize}
    \item \textbf{Gramáticas}: se guardan en formato XML \cite{xml} en la ubicación que elija el usuario.
    \item \textbf{Informes PDF}: se generan en la ubicación seleccionada por el usuario usando iText \cite{itextpdf}.
    \item \textbf{Archivos de configuración}: se almacenan temporalmente en memoria durante la sesión.
    \item \textbf{Archivos de ayuda}: incluidos en la aplicación, no requieren instalación adicional.
\end{itemize}

\section{Desinstalación}

\subsection{Desinstalación en macOS}

Para desinstalar SimAS 3.0 en macOS:

\begin{enumerate}
    \item Abra la carpeta \texttt{Aplicaciones}.
    \item Arrastre \texttt{SimAS.app} a la papelera.
    \item No es necesario eliminar carpetas adicionales, ya que SimAS 3.0 no crea archivos de configuración permanentes.
\end{enumerate}

\subsection{Desinstalación en Windows}

Para desinstalar SimAS 3.0 en Windows:

\begin{enumerate}
    \item Abra el Panel de control.
    \item Seleccione \string"Programas y características\string".
    \item Busque \string"SimAS 3.0\string" en la lista.
    \item Haga clic en \string"Desinstalar\string" y siga las instrucciones.
    \item No es necesario eliminar carpetas adicionales, ya que SimAS 3.0 no crea archivos de configuración permanentes.
\end{enumerate}

\subsection{Desinstalación en Linux}

Para desinstalar SimAS 3.0 en Linux:

\begin{enumerate}
    \item Elimine la carpeta de instalación: \texttt{sudo rm -rf /opt/simas}.
    \item Elimine los accesos directos creados.
    \item No es necesario eliminar carpetas adicionales, ya que SimAS 3.0 no crea archivos de configuración permanentes.
\end{enumerate}

\section{Solución de problemas}

\subsection{Problemas comunes}

Si experimenta problemas durante la instalación o ejecución:

\begin{itemize}
    \item \textbf{Error \string"Java no encontrado\string"}: verifique que Java 17+ esté instalado correctamente.
    \item \textbf{Aplicación no se abre}: compruebe los permisos de ejecución del archivo.
    \item \textbf{Error de memoria}: cierre otras aplicaciones para liberar memoria RAM.
    \item \textbf{Interfaz no se muestra correctamente}: verifique la resolución de pantalla.
\end{itemize}

\subsection{Información para soporte técnico}

Si necesita ayuda técnica, proporcione la siguiente información:

\begin{enumerate}
    \item Sistema operativo y versión.
    \item Versión de Java instalada (ejecute \texttt{java -version}).
    \item Descripción detallada del problema.
    \item Pasos para reproducir el error.
\end{enumerate}

\section{Actualizaciones futuras}

\subsection{Política de versiones}

SimAS 3.0 es una versión estable y completa. Las futuras mejoras y nuevas funcionalidades se desarrollarán como nuevas versiones principales:

\begin{itemize}
    \item \textbf{SimAS 4.0}: futuras mejoras y nuevas características.
    \item \textbf{SimAS 5.0}: versiones posteriores con funcionalidades avanzadas.
    \item \textbf{SimAS 6.0}: y así sucesivamente.
\end{itemize}

\subsection{Información sobre nuevas versiones}

Para estar informado sobre nuevas versiones:

\begin{enumerate}
    \item Consulte el sitio web oficial del proyecto.
    \item Revise las publicaciones académicas relacionadas.
    \item Contacte con el desarrollador para información sobre futuras versiones.
\end{enumerate}

\subsection{Migración entre versiones}

Al instalar una nueva versión principal:

\begin{enumerate}
    \item Sus gramáticas guardadas serán compatibles con versiones futuras.
    \item Los informes PDF generados se mantendrán accesibles.
    \item No será necesario migrar configuraciones, ya que SimAS no utiliza archivos de configuración permanentes.
\end{enumerate}
