\chapter{Objetivos} \label{cap:objetivos}
\section{Objetivo principal}

El objetivo principal de este Trabajo de Fin de Grado es desarrollar la aplicación informática ``SimAS 3.0 descendente predictivo'', que permita la simulación de analizadores sintácticos descendentes predictivos, así como la generación de árboles sintácticos asociados a sus derivaciones sintácticas.

\section{Objetivos específicos}

El objetivo principal se desglosa en los siguientes objetivos específicos:

\begin{itemize}
    \item Revisar la versión anterior de SimAS 2.0 \cite{juan} para identificar y corregir errores y deficiencias del análisis sintáctico descendente predictivo,  tales como:
    \begin{itemize}
        \item Mejorar la edición  y el almacenamiento de gramáticas de contexto libre.
        \item Depurar la validación automática de gramáticas de contexto libre: comprobar que la gramática no tiene símbolos inútiles.
        \item Permitir la visualización de las gramáticas originales y transformadas.
        \item Definir funciones para el tratamiento automático de errores.
        \item Garantizar la simulación correcta de gramáticas que incluyan reglas épsilon.
        \item Generar correctamente los conjuntos Primero y Siguiente.
        \item Asegurar la generación adecuada de informes en formato PDF.
    \end{itemize}
    \item Ampliar las capacidades de SimAS para incluir la generación de árboles sintácticos de las derivaciones producidas por los analizadores sintácticos descendente predictivos:
    \begin{itemize}
        \item Permitir la generación paso a paso de los árboles sintácticos asociados a las derivaciones.
        \item Facilitar la exportación de los árboles sintácticos generados en diversos formatos.
    \end{itemize}
\end{itemize}


\section{Objetivos personales}

En el ámbito personal y profesional, se plantean los siguientes objetivos a alcanzar durante el desarrollo de este Trabajo de Fin de Grado:

\begin{itemize}
    \item \textbf{Contribuir a la educación en informática:} desarrollar una aplicación informática que no solo cumpla con los requisitos académicos del proyecto, sino que también pueda utilizarse como herramienta educativa en el ámbito de la enseñanza de los analizadores sintácticos. Se busca que esta aplicación sea útil para estudiantes y profesores en el aprendizaje y la enseñanza de este tema fundamental.
    
    \item \textbf{Profundizar en Java y Java Swing:} aprovechar este proyecto como una oportunidad para mejorar mis habilidades en el lenguaje de programación Java y en el uso de sus bibliotecas, especialmente Java Swing \cite{javaswing}. Se busca obtener un conocimiento más sólido y práctico en el desarrollo de aplicaciones gráficas de usuario, lo que puede ser beneficioso para futuros proyectos y oportunidades profesionales.

    \item \textbf{Explorar otras herramientas de desarrollo:} esta tarea implica investigar y adquirir conocimientos sobre diversas herramientas y tecnologías utilizadas en el desarrollo de aplicaciones informáticas, además de Java y Java Swing. Entre estas herramientas, un ejemplo destacado es Scene Builder \cite{scenebuilder}. Scene Builder es una herramienta gráfica que permite diseñar interfaces de usuario de manera intuitiva y visual, utilizando la tecnología JavaFX. Permite arrastrar y soltar componentes de la interfaz gráfica, configurar propiedades y establecer el diseño de la aplicación de forma rápida y eficiente. Explorar y familiarizarse con herramientas como Scene Builder amplía el repertorio de herramientas disponibles para el desarrollo de interfaces de usuario, lo que permite evaluar y seleccionar las mejores soluciones para proyectos futuros.
    
    \item \textbf{Aprender sobre el ciclo de desarrollo de software:} obtener una comprensión más profunda del proceso de desarrollo de software, desde la planificación y el diseño hasta la implementación, las pruebas y la documentación. Se busca adquirir experiencia práctica en todas las etapas del ciclo de vida del desarrollo de software y aprender a aplicar metodologías y buenas prácticas de ingeniería de software en un proyecto real.
\end{itemize}