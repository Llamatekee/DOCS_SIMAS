\chapter{Futuras mejoras} \label{cap: mejoras}
%%%%%%%%%%%%%%%%%%%%%%%%%%
% Permitir que la aplicación pueda usar varias gramáticas simultáneamente.


%%%%%%%%%%%%%%%%%%%%


%El usuario podrá guardar estas funciones de error para utilizarlas en varias gramáticas .

\section{Introducción}

% En este capítulo se enumeran las futuras mejoras que podrían realizarse a SimAS. Estas mejoras permitirían mejorar su funcionamiento y harían que el programa fuera más completo.

% SimAS es un programa que ha cubierto todos los objetivos pro\-pues\-tos, 
% siendo, además, muy intuitivo y amigable con el usuario. Sin embargo, hay ciertos aspectos que podrían ser mejorados o añadidos en un futuro. 

% Las mejoras que se podrían hacer al programa son las siguientes:

% \begin{itemize}
%  \item Internacionalización.
%  \item Respetar el orden de las reglas.
%  \item Generación de \textit{grafos} del AFD.
% \item Utilizar varias gramáticas simultáneamente.
% \item Traducciones basadas en la sintaxis o esquemas de traducción.
% \end{itemize}

\section{Internacionalización}

% Sería deseable que la aplicación estuviera disponible en otros idiomas. Esto permitiría potenciar al máximo su ámbito didáctico, puesto que SimAS podría ser utilizado en cualquier universidad donde el estudio de los diferentes métodos del análisis sintáctico estuviera dentro de los planes de estudio.

% Las traducciones de la aplicación serían guardadas en el paquete de recursos y se cargaría la aplicación con un idioma u otro, según las preferencias del usuario (almacenadas en el fichero de configuración de la aplicación).


 \section{Respetar orden de las reglas}

% Al cargar o modificar una gramática, la aplicación debería ordenar sus reglas de producción en función del símbolo no terminal de la parte izquierda. Esta mejora permitiría visualizar la gramática de una forma más correcta, independientemente de cómo fueran introducidas o modificadas sus reglas de producción.

 \section{Generación de grafos del AFD}

% Los métodos de análisis sintáctico ascendente LR utilizan un autómata finito determinista (AFD) para generar su tabla de análisis sintáctico. Actualmente dicho AFD se muestra de forma tabular. La mejora que se propone es que dicho autómata se muestre también mediante un grafo dirigido.

 \section{Utilizar varias gramáticas simultáneamente}

% Esta mejora permitiría al usuario poder editar y simular múltiples gramáticas a la misma vez, de manera que, si el usuario desea comparar de alguna manera las gramáticas, le sea mucho más fácil. Esta mejora convertiría a SimAS en un aplicación MDI: \textit{multiple document interface}.


 \section{Traducciones basadas en la sintaxis}

% Esta mejora permitiría incluir también el estudio y simulación del análisis semántico en SimAS. Mediante la realización de esta mejora, el software SimAS se completaría, permitiendo al alumno 
% estudiar todo el proceso de análisis de un programa, que está formado por las fases de análisis \textit{léxico}, \textit{sintáctico}
% y \textit{semántico}.

 \section{Algoritmos de Símbolos útiles, generadores y accesibles}

% Esta mejora permitiría incluir también los algoritmos de símbolos útiles, generadores y accesibles, para mejorar la validación de la gramática.
