\chapter{Recursos}

En este capítulo se expondrán los recursos usados durante la realización de este proyecto. Los recursos se definen como aquellos medios de los que se dispone para abordar el proceso de desarrollo del proyecto.

Existen tres grupos de recursos:

 \begin{itemize}
     \item Recursos humanos.
     \item Recursos de hardware.
     \item Recursos de software.
 \end{itemize}


 \section{Recursos humanos}
 \begin{itemize}
     \item \textbf{Autor}
     \item[] Antonio Llamas García, estudiante del Grado en Ingeniería Informática, especialidad en Computación.
    
     \item \textbf{Director}
     \item[] Dr. Nicolás Luis Fernández García. Profesor Titular de Universidad del área de Ciencia de la Computación e Inteligencia Artificial, perteneciente al Departamento de Informática y Análisis Numérico de la Universidad de Córdoba.
 \end{itemize}

\section{Recursos de hardware}
 Durante el desarrollo del proyecto se empleará el equipo informático a disposición del autor, cuya descripción corresponde con:
  \begin{itemize}
      \item Ordenador Portátil Características:
      \begin{itemize}
         \item Procesador Intel(R) Core(TM) 2,4 GHz i5 de 4 núcleos
         \item 16 Gbytes de memoria RAM.
         \item Intel Iris Plus Graphics 655 1536 MB.
         \item Almacenamiento: 240 Gb SSD.
      \end{itemize}
  \end{itemize}

\section{Recursos de software}
Se empleará Scene Builder \cite{scenebuilder} junto con JavaFX \cite{javafx} para, el lugar de diseñar la interfaz manualmente programando, realizar este cometido de una forma mucho más intuitiva. Scene Builder es una herramienta que permite diseñar interfaces gráficas de manera visual para aplicaciones JavaFX. Esta elección se debe al hecho de que Scene Builder trabaja con JavaFX, lo que facilitará una creación más intuitiva y eficiente de la interfaz de usuario, mejorando así el proceso de desarrollo.

Además, se modularizarán las características de los elementos de los ficheros XML de gramáticas, permitiendo una separación más clara y una gestión más eficiente de los datos. Finalmente, se mejorará el uso de Graphviz \cite{graphviz}, una biblioteca utilizada en la versión anterior, para permitir la generación de los árboles sintácticos de forma iterativa, mejorando así la visualización y comprensión de los resultados del análisis sintáctico.

 \subsection{Sistema Operativos}
 \begin{itemize}
      \item MacOs Sonoma, versión 14.0
  \end{itemize}

 
  \subsection{Entorno de desarrollo y lenguaje de programación}
  \begin{itemize}
      \item IntelliJ IDEA\cite{intellij}: entorno de desarrollo utilizado para la codificación y pruebas de la aplicación.
      \item Java \cite{java}: lenguaje de programación multiplataforma.
      \item Java Swing \cite{javaswing}: biblioteca de Java para el desarrollo de aplicaciones gráficas.
  \end{itemize}

 \subsection{Herramientas de documentación}
  \begin{itemize}
      \item \LaTeX \ \cite{latex}: software de formateo de documentación científica y técnica.
      \item OverLeaf \cite{overleaf}: \textit{front-end} web para \LaTeX.
  \end{itemize}

  \subsection{Herramientas para la edición de diagramas}
  \begin{itemize}
      \item Draw.io (Diagrams.net) \cite{drawio}: editor gráfico para el diseño de los diagramas.
  \end{itemize}
