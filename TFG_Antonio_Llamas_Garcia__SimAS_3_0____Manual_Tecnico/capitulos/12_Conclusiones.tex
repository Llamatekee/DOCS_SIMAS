\chapter{Conclusiones}

% La principal conclusión del presente proyecto de fin de carrera es que se ha conseguido corregir y mejorar la aplicación de escritorio denominada \textbf{SimAS} de manera que permita generar cualquier tipo de análisis sintáctico, tanto descendente como ascendente. Con ella, los alumnos que cursen la asignatura \textit{Procesadores del Lenguaje} tendrán una herramienta que les ayudará en la comprensión del análisis sintáctico de gramáticas de contexto libre.

% \textbf{SimAS} permitía anteriormente:
% \begin{enumerate}
% \item La \textit{edición} de gramáticas de contexto libre, permitiendo la definición de los símbolos terminales, no terminales, símbolo inicial y reglas de producción.
% \item El \textit{almacenamiento} y \textit{recuperación} de gramáticas de contexto libre utilizando ficheros con formato XML.
% \item La \textit{generación} de un informe de la gramática, que contiene una representación de la gramática, así como una especificación de todos los componentes que la forman.
% \item La \textit{generación} de los conjuntos auxiliares \textit{primero} y \textit{siguiente}.
% \item La \textit{generación} de las colecciones canónicas de elementos LR(0), de elementos LR(1) y de elementos LALR(1).
% \item La \textit{construcción} de las tablas de análisis sintáctico descendente (tabla predictiva) y ascendente (SLR, LR-canónico y LALR).
% \item La \textit{definición} de funciones de error asociadas al método de recuperación de nivel de frase.
% \item La \textit{simulación} de los métodos de análisis sintáctico descendente y ascendente (SLR, LR-canónico y LALR) utilizando las tablas y las funciones de error.
% \item La \textit{consulta} de un manual de ayuda que permite resolver las dudas sobre el funcionamiento de la aplicación.
% \end{enumerate}

% Actualmente, además permite también
% \begin{enumerate}
%     \item Mayor \textit{modularización} de las gramáticas en los ficheros XML para mejorar la comprensión del alumno.
%     \item La \textit{validación} automática al carga una gramática.
%     \item La \textit{corrección} de los errores que suponía la recursividad y factorización por la izquierda junto con las reglas épsilon. 
%     \item La \textit{definición} automática de funciones de error predefinidas.
%     \item La \textit{visualización} de la gramática a partir de la cual se están generando las tablas, conjuntos y funciones de error.
%     \item La \textit{generación} de árboles sintácticos de forma paralela a la simulación.
%     \item La \textit{generación} de la derivación correspondiente y de forma paralela a la simulación.
%     \item La correcta \textit{generación} de un informe que muestra los pasos realizados en la simulación.
% \end{enumerate}

% Se ha creado una interfaz gráfica sencilla e intuitiva que permite a los usuarios centrarse en el contenido didáctico de la aplicación, sin gastar demasiado
% tiempo en la comprensión o aprendizaje de la interfaz de la misma. Todo esto, integrado con los módulos de la aplicación, ha hecho que se cumplan los objetivos didácticos de forma satisfactoria. Cumpliendo este tipo de objetivos,  se consigue que la aplicación pueda llevar a cabo su labor didáctica de forma satisfactoria.

% En conclusión, \textbf{SimAS} ahora es una herramienta con más funcionalidades y ayudas para el alumnado de manera que se le permita comprender mejor la asignatura. Se puede considerar que se han cumplido todos los objetivos planteados y que cumplirá
% perfectamente su labor de enseñanza de los métodos sintácticos, tanto descendente como ascendente, 
% permitiendo una mejora en la comprensión de estos métodos en los cursos futuros donde se imparta esta docencia.
