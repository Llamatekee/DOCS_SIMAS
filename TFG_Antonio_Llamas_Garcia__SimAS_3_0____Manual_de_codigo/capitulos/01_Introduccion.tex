\chapter{Introducción}\label{cap-introduccion}

\section{Propósito del documento}

Este manual de código tiene como objetivo proporcionar una documentación técnica completa y detallada de la aplicación \textbf{SimAS 3.0} (Simulador de Análisis Sintáctico), desarrollada como Trabajo de Fin de Grado en Ingeniería Informática en la Universidad de Córdoba.

El documento está dirigido a desarrolladores, mantenedores del software y cualquier persona que necesite comprender la estructura interna, el funcionamiento y los algoritmos implementados en la aplicación. A diferencia de versiones anteriores del manual, este documento no incluye código fuente embebido, sino que hace referencia al repositorio oficial donde se encuentra todo el código fuente completo y actualizado.

\section{Descripción general del proyecto}

SimAS 3.0 es una aplicación educativa desarrollada en Java que implementa un simulador de analizadores sintácticos descendentes predictivos. La aplicación permite a los usuarios:

\begin{itemize}
    \item Crear y editar gramáticas libres de contexto
    \item Generar automáticamente tablas de análisis predictivo
    \item Simular el proceso de análisis sintáctico descendente
    \item Visualizar el árbol de derivación resultante
    \item Gestionar funciones de error personalizadas
\end{itemize}

\section{Repositorio oficial del proyecto}

Todo el código fuente de SimAS 3.0 está disponible en el repositorio oficial de GitHub:

\begin{center}
\url{https://github.com/Llamatekee/SimAS-3.0.git}
\end{center}

\subsection{Estructura del repositorio}

El repositorio está organizado de la siguiente manera:

\begin{itemize}
    \item \texttt{src/}: Directorio principal del código fuente
    \item \texttt{lib/}: Librerías externas (JavaFX SDK)
    \item \texttt{build/}: Scripts de construcción
    \item \texttt{dist-standalone/}: Ejecutables generados
    \item \texttt{resources/}: Recursos gráficos e internacionalización
    \item \texttt{vistas/}: Archivos FXML de la interfaz
\end{itemize}

\subsection{Acceso al código fuente}

Para acceder al código fuente completo:

\begin{enumerate}
    \item Clonar el repositorio:
    \begin{lstlisting}
    git clone https://github.com/Llamatekee/SimAS-3.0.git
    \end{lstlisting}

    \item Explorar la estructura de paquetes en \texttt{src/}
    \item Revisar la documentación de clases en los archivos Java
    \item Consultar los archivos de configuración y scripts de construcción
\end{enumerate}

\section{Características principales}

\subsection{Interfaz de usuario moderna}
La aplicación utiliza JavaFX 17 para proporcionar una interfaz de usuario moderna e intuitiva, con:
\begin{itemize}
    \item Diseño responsivo y adaptable
    \item Sistema de pestañas para múltiples proyectos
    \item Interfaz completamente internacionalizada (6 idiomas)
    \item Atajos de teclado para operaciones frecuentes
\end{itemize}

\subsection{Algoritmos de análisis sintáctico}
Implementa algoritmos estándar de análisis sintáctico descendente predictivo:
\begin{itemize}
    \item Construcción de conjuntos FIRST y FOLLOW
    \item Generación de tablas de análisis predictivo
    \item Algoritmo de análisis LL(1)
    \item Detección y manejo de conflictos
    \item Funciones de error personalizables
\end{itemize}

\subsection{Arquitectura modular}
El sistema está diseñado con una arquitectura modular que separa claramente:
\begin{itemize}
    \item Lógica de negocio (modelo de gramáticas)
    \item Interfaz de usuario (controladores y vistas)
    \item Algoritmos de análisis (simulador)
    \item Utilidades y servicios auxiliares
\end{itemize}

\section{Alcance del manual}

Este manual cubre los siguientes aspectos de la aplicación:

\begin{enumerate}
    \item \textbf{Acceso al repositorio}: cómo obtener y navegar el código fuente.
    \item \textbf{Arquitectura del sistema}: diseño general y organización de componentes.
    \item \textbf{Documentación de paquetes}: descripción detallada de cada paquete Java.
    \item \textbf{Clases principales}: documentación de las clases más importantes.
    \item \textbf{Algoritmos implementados}: explicación de los algoritmos de análisis sintáctico.
    \item \textbf{Interfaz de usuario}: documentación de la capa de presentación.
    \item \textbf{Sistema de internacionalización}: gestión de múltiples idiomas.
    \item \textbf{Compilación y despliegue}: proceso de construcción de la aplicación.
    \item \textbf{Convenciones y mejores prácticas}: estándares de desarrollo utilizados.
\end{enumerate}

\section{Convenciones utilizadas}

A lo largo de este manual se utilizarán las siguientes convenciones:

\begin{itemize}
    \item \textbf{Nombres de clases}: se escribirán en formato \texttt{ClaseNombre}.
    \item \textbf{Nombres de métodos}: se escribirán en formato \texttt{metodoNombre()}.
    \item \textbf{Paquetes}: se escribirán en formato \texttt{paquete.subpaquete}.
    \item \textbf{Archivos}: se escribirán en formato \texttt{archivo.extensión}.
    \item \textbf{Rutas del repositorio}: referencias a archivos usando rutas relativas al repositorio.
    \item \textbf{Enlaces a GitHub}: enlaces directos a archivos específicos en el repositorio.
\end{itemize}

\section{Estructura del documento}

El manual está organizado en las siguientes secciones principales:

\begin{description}
    \item[Parte I - Acceso y Arquitectura] Contiene información sobre el repositorio, estructura general y arquitectura del sistema.
    \item[Parte II - Componentes del Sistema] Incluye la documentación detallada de paquetes, clases y algoritmos.
    \item[Parte III - Desarrollo y Despliegue] Cubre aspectos técnicos de desarrollo, internacionalización y despliegue.
\end{description}

Esta estructura permite tanto una visión general del sistema como referencias específicas al código fuente en el repositorio, facilitando la comprensión y mantenimiento del código.

\section{Referencias técnicas}

Este manual se basa en los principios fundamentales de análisis sintáctico descritos en \cite{aho2006compilers} y \cite{hopcroft2006introduction}, implementados utilizando las tecnologías Java \cite{java} y JavaFX \cite{javafx}. La aplicación utiliza herramientas modernas de desarrollo como \cite{jpackage} para la generación de ejecutables nativos.

Para información detallada sobre la implementación específica de algoritmos y estructuras de datos, consulte directamente el código fuente en el repositorio oficial.

\textbf{Referencias al Manual Técnico:} Para explicaciones más detalladas sobre el diseño arquitectónico, análisis de clases, métricas de complejidad y patrones de diseño implementados en SimAS 3.0, consulte:
\begin{itemize}
    \item Capítulo 8: "Diseño de paquetes" - Arquitectura completa del sistema
    \item Capítulo 9: "Diseño de clases" - Implementación detallada de todas las clases
\end{itemize}