\chapter{Introducción}\label{cap-introduccion}

\section{Propósito del documento}

Este manual de código tiene como objetivo proporcionar una documentación técnica completa y detallada de la aplicación \textbf{SimAS 3.0} (Simulador de Análisis Sintáctico), desarrollada como Trabajo de Fin de Grado en Ingeniería Informática en la Universidad de Córdoba.

El documento está dirigido a desarrolladores, mantenedores del software y cualquier persona que necesite comprender la estructura interna, el funcionamiento y los algoritmos implementados en la aplicación.

\section{Descripción general del proyecto}

SimAS 3.0 es una aplicación educativa desarrollada en Java que implementa un simulador de analizadores sintácticos descendentes predictivos. La aplicación permite a los usuarios:

\begin{itemize}
    \item Crear y editar gramáticas libres de contexto
    \item Generar automáticamente tablas de análisis predictivo
    \item Simular el proceso de análisis sintáctico descendente
    \item Visualizar el árbol de derivación resultante
    \item Gestionar funciones de error personalizadas
\end{itemize}

\section{Características principales}

\subsection{Interfaz de usuario moderna}
La aplicación utiliza JavaFX 17 para proporcionar una interfaz de usuario moderna e intuitiva, con:
\begin{itemize}
    \item Diseño responsivo y adaptable
    \item Sistema de pestañas para múltiples proyectos
    \item Interfaz completamente internacionalizada
    \item Atajos de teclado para operaciones frecuentes
\end{itemize}

\subsection{Algoritmos de análisis sintáctico}
Implementa algoritmos estándar de análisis sintáctico descendente predictivo:
\begin{itemize}
    \item Construcción de conjuntos FIRST y FOLLOW
    \item Generación de tablas de análisis predictivo
    \item Algoritmo de análisis LL(1)
    \item Detección y manejo de conflictos
\end{itemize}

\subsection{Arquitectura modular}
El sistema está diseñado con una arquitectura modular que separa claramente:
\begin{itemize}
    \item Lógica de negocio (modelo de gramáticas)
    \item Interfaz de usuario (controladores y vistas)
    \item Algoritmos de análisis (simulador)
    \item Utilidades y servicios auxiliares
\end{itemize}

\section{Alcance del manual}

Este manual cubre los siguientes aspectos de la aplicación:

\begin{enumerate}
    \item \textbf{Arquitectura del sistema}: Diseño general y organización de componentes
    \item \textbf{Documentación de paquetes}: Descripción detallada de cada paquete Java
    \item \textbf{Clases principales}: Documentación de las clases más importantes
    \item \textbf{Algoritmos implementados}: Explicación de los algoritmos de análisis sintáctico
    \item \textbf{Interfaz de usuario}: Documentación de la capa de presentación
    \item \textbf{Sistema de internacionalización}: Gestión de múltiples idiomas
    \item \textbf{Compilación y despliegue}: Proceso de construcción de la aplicación
\end{enumerate}

\section{Convenciones utilizadas}

A lo largo de este manual se utilizarán las siguientes convenciones:

\begin{itemize}
    \item \textbf{Código Java}: Se mostrará con sintaxis resaltada y numeración de líneas
    \item \textbf{Nombres de clases}: Se escribirán en formato \texttt{ClaseNombre}
    \item \textbf{Nombres de métodos}: Se escribirán en formato \texttt{metodoNombre()}
    \item \textbf{Paquetes}: Se escribirán en formato \texttt{paquete.subpaquete}
    \item \textbf{Archivos}: Se escribirán en formato \texttt{archivo.extensión}
\end{itemize}

\section{Estructura del documento}

El manual está organizado en las siguientes secciones principales:

\begin{description}
    \item[Parte I - Documentación Externa] Contiene información sobre recursos, requisitos y arquitectura general del sistema.
    \item[Parte II - Documentación Interna] Incluye la documentación detallada del código fuente, algoritmos y componentes internos.
\end{description}

Esta estructura permite tanto una visión general del sistema como un análisis profundo de su implementación, facilitando la comprensión y mantenimiento del código.

\section{Referencias técnicas}

Este manual se basa en los principios fundamentales de análisis sintáctico descritos en \cite{aho2006compilers} y \cite{hopcroft2006introduction}, implementados utilizando las tecnologías Java \cite{java} y JavaFX \cite{javafx}. La aplicación utiliza herramientas modernas de desarrollo como \cite{jpackage} para la generación de ejecutables nativos.