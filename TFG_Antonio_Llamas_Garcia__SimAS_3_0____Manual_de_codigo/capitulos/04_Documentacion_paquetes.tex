\chapter{Documentación de Paquetes}\label{cap-documentacion-paquetes}

\section{Introducción}

Este capítulo proporciona una documentación detallada de cada paquete Java que constituye la aplicación SimAS 3.0. Para cada paquete se incluye una descripción de su propósito, las clases que contiene, las relaciones con otros paquetes y referencias directas al código fuente en el repositorio.

\textbf{Referencias al Manual Técnico:} Para explicaciones más detalladas sobre la arquitectura de paquetes, diagramas de dependencias completos, métricas de complejidad, análisis de acoplamiento/cohesión y patrones de diseño implementados, consulte:
\begin{itemize}
    \item Capítulo 8: "Diseño de paquetes" - Arquitectura completa y análisis detallado de dependencias.
    \item Capítulo 9: "Diseño de clases" - Implementación detallada de todas las clases del sistema.
\end{itemize}

La arquitectura de paquetes sigue principios de diseño orientado a objetos con separación clara de responsabilidades, bajo acoplamiento y alta cohesión. El sistema está organizado en 8 paquetes principales distribuidos en capas funcionales.

\section{Arquitectura general de paquetes}

SimAS 3.0 está estructurado siguiendo una \textit{arquitectura modular por capas} con separación clara de responsabilidades:

\begin{itemize}
    \item \textbf{Capa de Presentación} (\texttt{bienvenida, vistas}): gestiona la interfaz de usuario.
    \item \textbf{Capa de Lógica de Negocio} (\texttt{editor, simulador}): implementa la funcionalidad principal.
    \item \textbf{Capa de Modelo de Datos} (\texttt{gramatica}): representa las estructuras de datos fundamentales
    \item \textbf{Capa de Servicios Transversales} (\texttt{utils}): proporciona funcionalidades comunes.
    \item \textbf{Capa de Recursos} (\texttt{resources}): contiene recursos estáticos.
    \item \textbf{Capa de Documentación} (\texttt{centroayuda}): proporciona ayuda y documentación.
\end{itemize}

\textbf{Referencia a la arquitectura completa:} Consulte el Capítulo 8 del Manual Técnico para análisis detallado de la arquitectura de paquetes.

\section{Paquete bienvenida}

\subsection{Propósito}

El paquete \texttt{bienvenida} implementa la \textit{capa de presentación inicial} del sistema, actuando como punto de entrada principal de SimAS 3.0. Este paquete es fundamental para la experiencia del usuario, proporcionando una interfaz intuitiva y profesional.

\textbf{Ubicación en el repositorio:} \url{https://github.com/Llamatekee/SimAS-3.0/tree/main/src/bienvenida}

\textbf{Responsabilidades principales:}
\begin{itemize}
    \item Gestión completa del ciclo de vida inicial de la aplicación.
    \item Navegación centralizada hacia módulos funcionales.
    \item Coordinación entre componentes de la interfaz principal.
    \item Gestión de estados globales de la aplicación.
\end{itemize}

\subsection{Clases principales}

\subsubsection{Clase Bienvenida}

Clase principal que gestiona la pantalla de bienvenida de la aplicación.

\textbf{Ubicación:} \url{https://github.com/Llamatekee/SimAS-3.0/blob/main/src/bienvenida/Bienvenida.java}

\begin{itemize}
    \item \textbf{Herencia}: Extiende \texttt{Application} de JavaFX.
    \item \textbf{Patrón implementado}: Singleton para garantizar instancia única.
    \item \textbf{Responsabilidades}:
    \begin{itemize}
        \item Mostrar la pantalla de bienvenida con información del proyecto.
        \item Gestionar la transición automática al menú principal (2.5 segundos).
        \item Configurar el estilo y comportamiento de la ventana de bienvenida.
        \item Soporte completo de internacionalización.
    \end{itemize}
    \item \textbf{Métodos principales}:
    \begin{itemize}
        \item \texttt{start(Stage)}: inicializa la ventana de bienvenida
        \item \texttt{abrirMenuPrincipal()}: transita al menú principal
        \item \texttt{main(String[])}: punto de entrada estático
    \end{itemize}
\end{itemize}

\subsubsection{Clase MenuPrincipal}

Controlador principal que gestiona el menú principal de la aplicación y coordina la navegación.

\textbf{Ubicación:} \url{https://github.com/Llamatekee/SimAS-3.0/blob/main/src/bienvenida/MenuPrincipal.java}

\begin{itemize}
    \item \textbf{Herencia}: extiende \texttt{Application} de JavaFX.
    \item \textbf{Patrón implementado}: Facade para simplificar acceso a módulos complejos.
    \item \textbf{Responsabilidades}:
    \begin{itemize}
        \item Gestionar la interfaz del menú principal con sistema de pestañas avanzado.
        \item Coordinar la apertura de diferentes módulos (editor, simulador, ayuda).
        \item Implementar internacionalización completa con cambio dinámico de idioma.
        \item Gestionar sistema de ventanas secundarias y arrastrar/soltar pestañas.
        \item Configurar atajos de teclado para operaciones frecuentes.
        \item Gestionar navegación contextual y estados de la aplicación.
    \end{itemize}
    \item \textbf{Métodos principales}:
    \begin{itemize}
        \item \texttt{start(Stage)}: inicializa la ventana principal completa.
        \item \texttt{cambiarIdioma()}: gestiona cambio dinámico de idioma.
        \item \texttt{onBtnEditorAction()}: abre nueva instancia del editor.
        \item \texttt{onBtnSimuladorAction()}: carga gramática y abre simulador.
        \item \texttt{onBtnAyudaAction()}: abre documentación en navegador.
    \end{itemize}
\end{itemize}

\subsection{Dependencias y relaciones}

\begin{itemize}
    \item \textbf{Dependencias críticas}:
    \begin{itemize}
        \item \texttt{utils.*}: sistema de internacionalización, gestión de pestañas y ventanas (\textbf{85\% del sistema}).
        \item \texttt{resources}: iconos y recursos gráficos para la interfaz.
        \item \texttt{vistas}: archivos FXML para definición de interfaces.
    \end{itemize}
    \item \textbf{Relaciones con otros paquetes}:
    \begin{itemize}
        \item \texttt{editor}: crea instancias del editor de gramáticas.
        \item \texttt{simulador}: coordina lanzamiento del simulador descendente.
        \item \texttt{centroayuda}: integra sistema de ayuda contextual.
        \item \texttt{gramatica}: utiliza modelo de datos para gestión de gramáticas.
    \end{itemize}
\end{itemize}

\textbf{Características técnicas:}
\begin{itemize}
    \item \textbf{Complejidad}: 2 clases (5\% del total del sistema).
    \item \textbf{Patrones implementados}: Singleton, Facade, MVC (Vista-Controlador).
    \item \textbf{Responsabilidad}: punto de entrada único y navegación centralizada.
\end{itemize}

\section{Paquete editor}

\subsection{Propósito}

El paquete \texttt{editor} implementa el \textit{núcleo funcional de edición de gramáticas} de SimAS 3.0, proporcionando una interfaz completa para la creación, modificación y gestión de gramáticas de contexto libre. Este paquete representa el componente más complejo del sistema, con 10 clases que implementan un asistente paso a paso altamente sofisticado.

\textbf{Ubicación en el repositorio:} \url{https://github.com/Llamatekee/SimAS-3.0/tree/main/src/editor}

\textbf{Arquitectura y diseño:}

El paquete sigue una \textit{arquitectura modular jerárquica} organizada en tres niveles:
\begin{enumerate}
    \item \textbf{Nivel de Control Principal}: \texttt{Editor} y \texttt{EditorWindow} (gestión global).
    \item \textbf{Nivel de Coordinación}: \texttt{PanelCreacionGramatica} (orquestación del asistente).
    \item \textbf{Nivel de Especialización}: paneles específicos por funcionalidad.
\end{enumerate}

\subsection{Clases principales}

\subsubsection{Clase Editor}

Clase principal del editor que coordina todos los paneles de edición y gestiona el estado global.

\textbf{Ubicación:} \url{https://github.com/Llamatekee/SimAS-3.0/blob/main/src/editor/Editor.java}

\begin{itemize}
    \item \textbf{Herencia}: extiende \texttt{VBox} de JavaFX e implementa \texttt{ActualizableTextos}
    \item \textbf{Patrón implementado}: Mediator para coordinar comunicación entre paneles.
    \item \textbf{Responsabilidades}:
    \begin{itemize}
        \item Gestionar la gramática activa y su estado.
        \item Coordinar todos los paneles del asistente de creación.
        \item Implementar patrón Mediator para comunicación entre componentes.
        \item Gestionar persistencia de datos y validación global.
        \item Coordinar navegación entre pasos del asistente.
        \item Gestionar integración con el sistema de pestañas.
    \end{itemize}
    \item \textbf{Métodos principales}:
    \begin{itemize}
        \item \texttt{Editor(TabPane, MenuPrincipal)}: constructor con configuración completa.
        \item \texttt{cargarGramatica(Gramatica)}: carga gramática para edición.
        \item \texttt{validarGramaticaActual()}: validación completa de la gramática.
        \item \texttt{guardarGramatica()}: persistencia de datos.
    \end{itemize}
\end{itemize}

\subsubsection{EditorWindow.java}

Ventana principal del editor que contiene todos los paneles de edición.

\begin{itemize}
    \item \textbf{Responsabilidades}:
    \begin{itemize}
        \item Gestionar la interfaz de usuario del editor.
        \item Coordinar la navegación entre paneles.
        \item Gestionar el estado de la gramática en edición.
    \end{itemize}
\end{itemize}

\subsubsection{PanelCreacionGramatica.java}

Panel base que define la estructura común para todos los paneles de creación.

\subsubsection{PanelCreacionGramaticaPaso1.java}

Panel para la definición del nombre y descripción de la gramática.

\subsubsection{PanelCreacionGramaticaPaso2.java}

Panel para la definición de símbolos terminales y no terminales de la gramática.

\subsubsection{PanelCreacionGramaticaPaso3.java}

Panel para la definición de producciones de la gramática.

\subsubsection{PanelCreacionGramaticaPaso4.java}

Panel para la definición del símbolo inicial y validación de la gramática.

\subsubsection{PanelProducciones.java}

Panel especializado para la gestión de producciones con funcionalidades avanzadas.

\subsubsection{PanelSimbolosNoTerminales.java}

Panel especializado para la gestión de símbolos no terminales.

\subsubsection{PanelSimbolosTerminales.java}

Panel especializado para la gestión de símbolos terminales.

\subsection{Dependencias}

\begin{itemize}
    \item \texttt{gramatica.*}: para el modelo de datos de gramáticas.
    \item \texttt{utils.*}: para utilidades de interfaz y gestión de ventanas.
    \item \texttt{vistas.*}: para las definiciones FXML de la interfaz.
\end{itemize}

\section{Paquete gramatica}

\subsection{Propósito}

El paquete \texttt{gramatica} contiene el modelo de datos y los algoritmos fundamentales para el manejo de gramáticas libres de contexto y la generación de tablas de análisis predictivo.

\subsection{Clases principales}

\subsubsection{Gramatica.java}

Clase central que representa una gramática libre de contexto completa.

\begin{itemize}
    \item \textbf{Responsabilidades}:
    \begin{itemize}
        \item Almacenar todos los componentes de una gramática.
        \item Implementar algoritmos de análisis sintáctico.
        \item Generar tablas de análisis predictivo.
        \item Validar la gramática y detectar conflictos.
    \end{itemize}
    \item \textbf{Métodos principales}:
    \begin{itemize}
        \item \texttt{generarTablaPredictiva()}: genera la tabla de análisis predictivo.
        \item \texttt{calcularFirst()}: calcula los conjuntos PRIMERO.
        \item \texttt{calcularFollow()}: calcula los conjuntos SIGUIENTE.
        \item \texttt{esLL1()}: verifica si la gramática es LL(1).
    \end{itemize}
\end{itemize}

\subsubsection{Simbolo.java}

Clase abstracta base para todos los símbolos de la gramática.

\subsubsection{Terminal.java}

Representa un símbolo terminal de la gramática.

\subsubsection{NoTerminal.java}

Representa un símbolo no terminal de la gramática.

\subsubsection{Produccion.java}

Representa una producción de la gramática con su antecedente y consecuente.

\subsubsection{Antecedente.java}

Representa el lado izquierdo de una producción (no terminal).

\subsubsection{Consecuente.java}

Representa el lado derecho de una producción (secuencia de símbolos).

\subsubsection{TablaPredictiva.java}

Clase base para la representación de tablas de análisis predictivo.

\subsubsection{TablaPredictivaPaso5.java}

Implementación específica de la tabla predictiva con funcionalidades avanzadas.

\subsubsection{FilaTablaPredictiva.java}

Representa una fila de la tabla predictiva.

\subsubsection{FuncionError.java}

Representa una función de error personalizada para el análisis.

\subsection{Algoritmos implementados}

\subsubsection{Cálculo de conjuntos PRIMERO}

El algoritmo calcula para cada símbolo no terminal el conjunto de símbolos terminales que pueden aparecer al inicio de las cadenas derivadas.

\subsubsection{Cálculo de conjuntos SIGUIENTE}

El algoritmo calcula para cada símbolo no terminal el conjunto de símbolos terminales que pueden aparecer inmediatamente después de él en alguna derivación.

\subsubsection{Generación de tabla predictiva}

Utiliza los conjuntos PRIMERO y SIGUIENTE para construir la tabla de análisis predictivo LL(1).

\subsection{Dependencias}

Este paquete es independiente y no tiene dependencias externas, sirviendo como base para otros paquetes.

\section{Paquete simulador}

\subsection{Propósito}

El paquete \texttt{simulador} implementa el simulador de análisis sintáctico descendente predictivo, permitiendo a los usuarios simular el proceso de análisis paso a paso.

\subsection{Clases principales}

\subsubsection{PanelSimuladorDesc.java}

Panel principal del simulador que coordina todos los componentes de simulación.

\subsubsection{PanelSimulacion.java}

Panel que gestiona la interfaz de usuario del simulador y el estado de la simulación.

\subsubsection{SimulacionFinal.java}

Clase que implementa la lógica de simulación del análisis sintáctico.

\subsubsection{PanelNuevaSimDescPaso*.java}

Conjunto de paneles para la configuración paso a paso de una nueva simulación.

\subsubsection{EditorCadenaEntradaController.java}

Controlador para la edición de cadenas de entrada a analizar.

\subsubsection{NuevaFuncionError.java}

Panel para la creación de nuevas funciones de error personalizadas.

\subsubsection{PanelGramaticaOriginal.java}

Panel que muestra la gramática original utilizada en la simulación.

\subsection{Características del simulador}

\begin{itemize}
    \item Simulación paso a paso del análisis sintáctico.
    \item Visualización de la pila de análisis.
    \item Visualización del estado de la entrada.
    \item Generación del árbol de derivación.
    \item Manejo de errores con funciones personalizadas.
    \item Interfaz intuitiva con controles de navegación.
\end{itemize}

\subsection{Dependencias}

\begin{itemize}
    \item \texttt{gramatica.*}: para acceso a gramáticas y tablas predictivas.
    \item \texttt{utils.*}: para utilidades de interfaz.
    \item \texttt{vistas.*}: para las definiciones FXML.
\end{itemize}

\section{Paquete utils}

\subsection{Propósito}

El paquete \texttt{utils} contiene utilidades generales, servicios de internacionalización y herramientas de gestión de ventanas que son utilizadas por toda la aplicación.

\subsection{Clases principales}

\subsubsection{SecondaryWindow.java}

Utilidad para la gestión de ventanas secundarias de la aplicación.

\subsubsection{TabManager.java}

Gestiona el sistema de pestañas para múltiples proyectos simultáneos.

\subsubsection{TabPaneMonitor.java}

Monitor que gestiona el estado y comportamiento de las pestañas.

\subsubsection{ActualizableTextos.java}

Interfaz para componentes que pueden actualizar sus textos según el idioma seleccionado.

\subsubsection{LanguageItem.java}

Representa un elemento de idioma en el sistema de internacionalización.

\subsubsection{LanguageListCell.java}

Celda personalizada para la visualización de idiomas en listas.

\subsection{Sistema de internacionalización}

El paquete incluye archivos de propiedades para múltiples idiomas:

\begin{itemize}
    \item \verb|messages_es.properties|: textos en español.
    \item \verb|messages_en.properties|: textos en inglés.
    \item \verb|messages_de.properties|: textos en alemán.
    \item \verb|messages_fr.properties|: textos en francés.
    \item \verb|messages_ja.properties|: textos en japonés.
    \item \verb|messages_pt.properties|: textos en portugués.
\end{itemize}

\subsection{Dependencias}

Este paquete es independiente y proporciona servicios a otros paquetes.

\section{Paquete centroayuda}

\subsection{Propósito}

El paquete \texttt{centroayuda} implementa el sistema de ayuda integrado de la aplicación, incluyendo documentación, tutoriales y información sobre el proyecto.

\subsection{Clases principales}

\subsubsection{AcercaDe.java}

Ventana que muestra información sobre la aplicación, desarrolladores y versión.

\subsection{Recursos incluidos}

\begin{itemize}
    \item \texttt{ayuda.html}: documentación principal de ayuda.
    \item \texttt{SimAS.html}: información específica sobre la aplicación.
    \item \verb|Tema_*.pdf|: documentos temáticos sobre análisis sintáctico.
    \item \texttt{imagenes/}: recursos gráficos para la documentación.
\end{itemize}

\subsection{Dependencias}

\begin{itemize}
    \item \texttt{utils.*}: para utilidades de interfaz.
\end{itemize}

\section{Paquete vistas}

\subsection{Propósito}

El paquete \texttt{vistas} contiene todos los archivos FXML que definen las interfaces de usuario de la aplicación, siguiendo el patrón de separación de vista y lógica.

\subsection{Archivos FXML principales}

\begin{itemize}
    \item \texttt{Bienvenida.fxml}: pantalla de bienvenida.
    \item \texttt{MenuPrincipal.fxml}: menú principal de la aplicación.
    \item \texttt{Editor.fxml}: interfaz principal del editor.
    \item \texttt{PanelCreacionGramaticaPaso*.fxml}: paneles de creación de gramáticas.
    \item \texttt{PanelSimulacion.fxml}: interfaz del simulador.
    \item \texttt{SimulacionFinal.fxml}: interfaz de simulación final.
    \item \texttt{styles2.css}: estilos CSS para la aplicación.
\end{itemize}

\subsection{Características de las vistas}

\begin{itemize}
    \item Diseño responsivo y adaptable.
    \item Uso de estilos CSS para personalización.
    \item Integración con el sistema de internacionalización.
    \item Separación clara entre presentación y lógica.
\end{itemize}

\section{Métricas y análisis de la arquitectura}

\subsection{Distribución de clases por paquete}

Con el conocimiento detallado de las 39 clases documentadas, podemos proporcionar métricas precisas sobre la complejidad del sistema:

\begin{table}[H]
\centering
\caption{Distribución de clases Java por paquete}
\label{tab:distribucion-clases}
\begin{tabular}{|l|c|c|}
\hline
\textbf{Paquete} & \textbf{Número de clases} & \textbf{Porcentaje} \\
\hline
simulador & 13 & 33\% \\
editor & 10 & 26\% \\
gramatica & 8 & 21\% \\
utils & 6 & 15\% \\
bienvenida & 2 & 5\% \\
\hline
\textbf{Total} & \textbf{39} & \textbf{100\%} \\
\hline
\end{tabular}
\end{table}

\subsection{Matriz de dependencias completa}

\begin{table}[H]
\centering
\caption{Matriz de dependencias entre paquetes}
\label{tab:matriz-dependencias}
\begin{tabular}{|l|c|c|c|c|c|c|c|c|}
\hline
\textbf{De $\rightarrow$ A} & bienv. & editor & sim. & gram. & utils & res. & centro & vistas \\
\hline
\hline
bienvenida & - & → & → & - & → & → & - & → \\
editor & - & - & - & → & → & → & → & → \\
simulador & - & - & - & → & → & → & → & → \\
gramatica & - & - & - & - & - & - & - & - \\
utils & - & - & - & - & - & - & - & - \\
resources & - & - & - & - & - & - & - & - \\
centroayuda & - & - & - & - & - & → & - & - \\
vistas & - & - & - & - & - & - & - & - \\
\hline
\end{tabular}
\end{table}

\subsection{Complejidad algorítmica por paquete}

\begin{table}[H]
\centering
\caption{Complejidad algorítmica de los paquetes principales}
\label{tab:complejidad-paquetes}
\begin{tabular}{|l|l|}
\hline
\textbf{Paquete} & \textbf{Complejidad algorítmica} \\
\hline
gramatica & O(n) a O(n³) (cálculo de conjuntos, construcción de tabla) \\
simulador & O(n³) (algoritmos de análisis sintáctico) \\
editor & O(n²) (validación y transformación de gramáticas) \\
utils & O(1) a O(n) (gestión de componentes y navegación) \\
bienvenida & O(1) (navegación básica) \\
\hline
\end{tabular}
\end{table}

\subsection{Análisis de acoplamiento y cohesión}

\begin{itemize}
    \item \textbf{Acoplamiento bajo}: las dependencias están claramente definidas y minimizadas.
    \item \textbf{Cohesión alta}: las clases dentro de cada paquete comparten fuerte relación funcional.
    \item \textbf{Punto único de fallo}: el paquete \texttt{gramatica} es crítico para todo el sistema.
    \item \textbf{Servicios transversales}: los paquetes \texttt{utils} y \texttt{resources} son utilizados por el 85\% del sistema.
\end{itemize}

\textbf{Referencias al repositorio:}

\begin{itemize}
    \item \textbf{Código fuente completo}: \url{https://github.com/Llamatekee/SimAS-3.0/tree/main/src}
    \item \textbf{Recursos del sistema}: \url{https://github.com/Llamatekee/SimAS-3.0/tree/main/src/resources}
    \item \textbf{Archivos de interfaz}: \url{https://github.com/Llamatekee/SimAS-3.0/tree/main/src/vistas}
    \item \textbf{Documentación técnica}: consulte Capítulo 8 del Manual Técnico para análisis detallado.
\end{itemize}
