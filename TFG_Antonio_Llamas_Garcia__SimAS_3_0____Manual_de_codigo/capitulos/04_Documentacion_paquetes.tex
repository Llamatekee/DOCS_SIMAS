\chapter{Documentación de Paquetes}\label{cap-documentacion-paquetes}

\section{Introducción}

Este capítulo proporciona una documentación detallada de cada paquete Java que constituye la aplicación SimAS 3.0. Para cada paquete se incluye una descripción de su propósito, las clases que contiene y las relaciones con otros paquetes.

\section{Paquete bienvenida}

\subsection{Propósito}

El paquete \texttt{bienvenida} se encarga de la gestión de la pantalla de bienvenida y el menú principal de la aplicación. Es el punto de entrada de la aplicación y coordina la navegación hacia los diferentes módulos.

\subsection{Clases principales}

\subsubsection{Bienvenida.java}

Clase principal que gestiona la pantalla de bienvenida de la aplicación.

\begin{itemize}
    \item \textbf{Herencia}: Extiende \texttt{Application} de JavaFX
    \item \textbf{Responsabilidades}:
    \begin{itemize}
        \item Mostrar la pantalla de bienvenida con información del proyecto
        \item Gestionar la transición automática al menú principal
        \item Configurar el estilo y comportamiento de la ventana de bienvenida
    \end{itemize}
    \item \textbf{Métodos principales}:
    \begin{itemize}
        \item \texttt{start(Stage primaryStage)}: Inicializa la ventana de bienvenida
        \item \texttt{abrirMenuPrincipal()}: Lanza el menú principal después del tiempo de espera
    \end{itemize}
\end{itemize}

\subsubsection{MenuPrincipal.java}

Controlador principal que gestiona el menú principal de la aplicación y coordina la navegación.

\begin{itemize}
    \item \textbf{Herencia}: Extiende \texttt{Application} de JavaFX
    \item \textbf{Responsabilidades}:
    \begin{itemize}
        \item Gestionar la interfaz del menú principal
        \item Coordinar la apertura de diferentes módulos (editor, simulador, ayuda)
        \item Gestionar el sistema de pestañas para múltiples proyectos
        \item Implementar la internacionalización de la interfaz
        \item Configurar atajos de teclado para operaciones frecuentes
    \end{itemize}
    \item \textbf{Métodos principales}:
    \begin{itemize}
        \item \texttt{onBtnEditorAction()}: Abre el editor de gramáticas
        \item \texttt{onBtnSimuladorAction()}: Abre el simulador
        \item \texttt{onBtnAyudaAction()}: Abre el sistema de ayuda
        \item \texttt{cambiarIdioma()}: Cambia el idioma de la interfaz
    \end{itemize}
\end{itemize}

\subsection{Dependencias}

\begin{itemize}
    \item \texttt{editor.Editor}: Para abrir el editor de gramáticas
    \item \texttt{simulador.PanelSimuladorDesc}: Para abrir el simulador
    \item \texttt{utils.*}: Para gestión de ventanas, internacionalización y tabs
    \item \texttt{gramatica.Gramatica}: Para manejo de datos de gramáticas
\end{itemize}

\section{Paquete editor}

\subsection{Propósito}

El paquete \texttt{editor} implementa el editor de gramáticas, permitiendo a los usuarios crear, modificar y validar gramáticas libres de contexto de manera visual e intuitiva.

\subsection{Clases principales}

\subsubsection{Editor.java}

Clase principal del editor que coordina todos los paneles de edición.

\begin{itemize}
    \item \textbf{Responsabilidades}:
    \begin{itemize}
        \item Coordinar los diferentes paneles de edición
        \item Gestionar el flujo de trabajo de creación de gramáticas
        \item Validar la gramática en cada paso del proceso
        \item Gestionar la persistencia de datos
    \end{itemize}
\end{itemize}

\subsubsection{EditorWindow.java}

Ventana principal del editor que contiene todos los paneles de edición.

\begin{itemize}
    \item \textbf{Responsabilidades}:
    \begin{itemize}
        \item Gestionar la interfaz de usuario del editor
        \item Coordinar la navegación entre paneles
        \item Gestionar el estado de la gramática en edición
    \end{itemize}
\end{itemize}

\subsubsection{PanelCreacionGramatica.java}

Panel base que define la estructura común para todos los paneles de creación.

\subsubsection{PanelCreacionGramaticaPaso1.java}

Panel para la definición del nombre y descripción de la gramática.

\subsubsection{PanelCreacionGramaticaPaso2.java}

Panel para la definición de símbolos terminales y no terminales de la gramática.

\subsubsection{PanelCreacionGramaticaPaso3.java}

Panel para la definición de producciones de la gramática.

\subsubsection{PanelCreacionGramaticaPaso4.java}

Panel para la definición del símbolo inicial y validación de la gramática.

\subsubsection{PanelProducciones.java}

Panel especializado para la gestión de producciones con funcionalidades avanzadas.

\subsubsection{PanelSimbolosNoTerminales.java}

Panel especializado para la gestión de símbolos no terminales.

\subsubsection{PanelSimbolosTerminales.java}

Panel especializado para la gestión de símbolos terminales.

\subsection{Dependencias}

\begin{itemize}
    \item \texttt{gramatica.*}: Para el modelo de datos de gramáticas
    \item \texttt{utils.*}: Para utilidades de interfaz y gestión de ventanas
    \item \texttt{vistas.*}: Para las definiciones FXML de la interfaz
\end{itemize}

\section{Paquete gramatica}

\subsection{Propósito}

El paquete \texttt{gramatica} contiene el modelo de datos y los algoritmos fundamentales para el manejo de gramáticas libres de contexto y la generación de tablas de análisis predictivo.

\subsection{Clases principales}

\subsubsection{Gramatica.java}

Clase central que representa una gramática libre de contexto completa.

\begin{itemize}
    \item \textbf{Responsabilidades}:
    \begin{itemize}
        \item Almacenar todos los componentes de una gramática
        \item Implementar algoritmos de análisis sintáctico
        \item Generar tablas de análisis predictivo
        \item Validar la gramática y detectar conflictos
    \end{itemize}
    \item \textbf{Métodos principales}:
    \begin{itemize}
        \item \texttt{generarTablaPredictiva()}: Genera la tabla de análisis predictivo
        \item \texttt{calcularFirst()}: Calcula los conjuntos FIRST
        \item \texttt{calcularFollow()}: Calcula los conjuntos FOLLOW
        \item \texttt{esLL1()}: Verifica si la gramática es LL(1)
    \end{itemize}
\end{itemize}

\subsubsection{Simbolo.java}

Clase abstracta base para todos los símbolos de la gramática.

\subsubsection{Terminal.java}

Representa un símbolo terminal de la gramática.

\subsubsection{NoTerminal.java}

Representa un símbolo no terminal de la gramática.

\subsubsection{Produccion.java}

Representa una producción de la gramática con su antecedente y consecuente.

\subsubsection{Antecedente.java}

Representa el lado izquierdo de una producción (no terminal).

\subsubsection{Consecuente.java}

Representa el lado derecho de una producción (secuencia de símbolos).

\subsubsection{TablaPredictiva.java}

Clase base para la representación de tablas de análisis predictivo.

\subsubsection{TablaPredictivaPaso5.java}

Implementación específica de la tabla predictiva con funcionalidades avanzadas.

\subsubsection{FilaTablaPredictiva.java}

Representa una fila de la tabla predictiva.

\subsubsection{FuncionError.java}

Representa una función de error personalizada para el análisis.

\subsection{Algoritmos implementados}

\subsubsection{Cálculo de conjuntos FIRST}

El algoritmo calcula para cada símbolo no terminal el conjunto de símbolos terminales que pueden aparecer al inicio de las cadenas derivadas.

\subsubsection{Cálculo de conjuntos FOLLOW}

El algoritmo calcula para cada símbolo no terminal el conjunto de símbolos terminales que pueden aparecer inmediatamente después de él en alguna derivación.

\subsubsection{Generación de tabla predictiva}

Utiliza los conjuntos FIRST y FOLLOW para construir la tabla de análisis predictivo LL(1).

\subsection{Dependencias}

Este paquete es independiente y no tiene dependencias externas, sirviendo como base para otros paquetes.

\section{Paquete simulador}

\subsection{Propósito}

El paquete \texttt{simulador} implementa el simulador de análisis sintáctico descendente predictivo, permitiendo a los usuarios simular el proceso de análisis paso a paso.

\subsection{Clases principales}

\subsubsection{PanelSimuladorDesc.java}

Panel principal del simulador que coordina todos los componentes de simulación.

\subsubsection{PanelSimulacion.java}

Panel que gestiona la interfaz de usuario del simulador y el estado de la simulación.

\subsubsection{SimulacionFinal.java}

Clase que implementa la lógica de simulación del análisis sintáctico.

\subsubsection{PanelNuevaSimDescPaso*.java}

Conjunto de paneles para la configuración paso a paso de una nueva simulación.

\subsubsection{EditorCadenaEntradaController.java}

Controlador para la edición de cadenas de entrada a analizar.

\subsubsection{NuevaFuncionError.java}

Panel para la creación de nuevas funciones de error personalizadas.

\subsubsection{PanelGramaticaOriginal.java}

Panel que muestra la gramática original utilizada en la simulación.

\subsection{Características del simulador}

\begin{itemize}
    \item Simulación paso a paso del análisis sintáctico
    \item Visualización de la pila de análisis
    \item Visualización del estado de la entrada
    \item Generación del árbol de derivación
    \item Manejo de errores con funciones personalizadas
    \item Interfaz intuitiva con controles de navegación
\end{itemize}

\subsection{Dependencias}

\begin{itemize}
    \item \texttt{gramatica.*}: Para acceso a gramáticas y tablas predictivas
    \item \texttt{utils.*}: Para utilidades de interfaz
    \item \texttt{vistas.*}: Para las definiciones FXML
\end{itemize}

\section{Paquete utils}

\subsection{Propósito}

El paquete \texttt{utils} contiene utilidades generales, servicios de internacionalización y herramientas de gestión de ventanas que son utilizadas por toda la aplicación.

\subsection{Clases principales}

\subsubsection{SecondaryWindow.java}

Utilidad para la gestión de ventanas secundarias de la aplicación.

\subsubsection{TabManager.java}

Gestiona el sistema de pestañas para múltiples proyectos simultáneos.

\subsubsection{TabPaneMonitor.java}

Monitor que gestiona el estado y comportamiento de las pestañas.

\subsubsection{ActualizableTextos.java}

Interfaz para componentes que pueden actualizar sus textos según el idioma seleccionado.

\subsubsection{LanguageItem.java}

Representa un elemento de idioma en el sistema de internacionalización.

\subsubsection{LanguageListCell.java}

Celda personalizada para la visualización de idiomas en listas.

\subsection{Sistema de internacionalización}

El paquete incluye archivos de propiedades para múltiples idiomas:

\begin{itemize}
    \item \texttt{messages\_es.properties}: Textos en español
    \item \texttt{messages\_en.properties}: Textos en inglés
    \item \texttt{messages\_de.properties}: Textos en alemán
    \item \texttt{messages\_fr.properties}: Textos en francés
    \item \texttt{messages\_ja.properties}: Textos en japonés
    \item \texttt{messages\_pt.properties}: Textos en portugués
\end{itemize}

\subsection{Dependencias}

Este paquete es independiente y proporciona servicios a otros paquetes.

\section{Paquete centroayuda}

\subsection{Propósito}

El paquete \texttt{centroayuda} implementa el sistema de ayuda integrado de la aplicación, incluyendo documentación, tutoriales y información sobre el proyecto.

\subsection{Clases principales}

\subsubsection{AcercaDe.java}

Ventana que muestra información sobre la aplicación, desarrolladores y versión.

\subsection{Recursos incluidos}

\begin{itemize}
    \item \texttt{ayuda.html}: Documentación principal de ayuda
    \item \texttt{SimAS.html}: Información específica sobre la aplicación
    \item \texttt{Tema\_*.pdf}: Documentos temáticos sobre análisis sintáctico
    \item \texttt{imagenes/}: Recursos gráficos para la documentación
\end{itemize}

\subsection{Dependencias}

\begin{itemize}
    \item \texttt{utils.*}: Para utilidades de interfaz
\end{itemize}

\section{Paquete vistas}

\subsection{Propósito}

El paquete \texttt{vistas} contiene todos los archivos FXML que definen las interfaces de usuario de la aplicación, siguiendo el patrón de separación de vista y lógica.

\subsection{Archivos FXML principales}

\begin{itemize}
    \item \texttt{Bienvenida.fxml}: Pantalla de bienvenida
    \item \texttt{MenuPrincipal.fxml}: Menú principal de la aplicación
    \item \texttt{Editor.fxml}: Interfaz principal del editor
    \item \texttt{PanelCreacionGramaticaPaso*.fxml}: Paneles de creación de gramáticas
    \item \texttt{PanelSimulacion.fxml}: Interfaz del simulador
    \item \texttt{SimulacionFinal.fxml}: Interfaz de simulación final
    \item \texttt{styles2.css}: Estilos CSS para la aplicación
\end{itemize}

\subsection{Características de las vistas}

\begin{itemize}
    \item Diseño responsivo y adaptable
    \item Uso de estilos CSS para personalización
    \item Integración con el sistema de internacionalización
    \item Separación clara entre presentación y lógica
\end{itemize}

\section{Resumen de dependencias}

La siguiente tabla resume las dependencias principales entre paquetes:

\begin{table}[H]
\centering
\begin{tabular}{|l|l|}
\hline
\textbf{Paquete} & \textbf{Dependencias principales} \\
\hline
bienvenida & editor, simulador, utils, gramatica \\
\hline
editor & gramatica, utils, vistas \\
\hline
simulador & gramatica, utils, vistas \\
\hline
gramatica & (independiente) \\
\hline
utils & (independiente) \\
\hline
centroayuda & utils \\
\hline
vistas & utils \\
\hline
\end{tabular}
\caption{Dependencias entre paquetes de SimAS 3.0}
\label{tab:dependencias-paquetes}
\end{table}
