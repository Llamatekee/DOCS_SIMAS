\chapter{Sistema de Internacionalización}\label{cap-internacionalizacion}

\section{Introducción}

SimAS 3.0 implementa un sistema completo de internacionalización (i18n) que permite a la aplicación adaptarse a diferentes idiomas y regiones. Este sistema soporta 6 idiomas completos con conmutación en tiempo de ejecución, facilitando el acceso a usuarios de diferentes países y culturas.

\textbf{Ubicación en el repositorio:} \url{https://github.com/Llamatekee/SimAS-3.0/tree/main/src/utils}

Este capítulo documenta la arquitectura, implementación y uso del sistema de internacionalización, con referencias directas al código fuente en el repositorio.

\textbf{Referencias al Manual Técnico:} para explicaciones más detalladas sobre las clases, paquetes y dependencias del sistema de internacionalización, incluyendo diagramas de clases, análisis de complejidad y patrones de diseño implementados, consulte:
\begin{itemize}
    \item Capítulo 8: "Diseño de paquetes" - Arquitectura general y dependencias.
    \item Capítulo 9: "Diseño de clases" - Implementación detallada de clases como LanguageItem y LanguageListCell.
\end{itemize}

\section{Arquitectura del sistema}

\subsection{Componentes principales}

El sistema de internacionalización está compuesto por los siguientes elementos:

\begin{itemize}
    \item \textbf{Archivos de propiedades}: contienen las traducciones para cada idioma (ubicados en \texttt{src/utils/}).
    \item \textbf{LanguageItem}: representa un idioma disponible con su configuración completa.
    \item \textbf{LanguageListCell}: celda personalizada para mostrar idiomas en interfaces.
    \item \textbf{ActualizableTextos}: interfaz para componentes que pueden actualizar sus textos.
    \item \textbf{ResourceBundle}: mecanismo de Java para cargar recursos localizados.
    \item \textbf{Internacionalización reactiva}: sistema de actualización automática de textos.
\end{itemize}

\subsection{Idiomas soportados}

La aplicación soporta los siguientes idiomas:

\begin{table}[H]
\centering
\begin{tabular}{|l|l|l|}
\hline
\textbf{Código} & \textbf{Idioma} & \textbf{Archivo} \\
\hline
es & Español & messages\_es.properties \\
\hline
en & English & messages\_en.properties \\
\hline
de & Deutsch & messages\_de.properties \\
\hline
fr & Français & messages\_fr.properties \\
\hline
ja & Japanese & messages\_ja.properties \\
\hline
pt & Português & messages\_pt.properties \\
\hline
\end{tabular}
\caption{Idiomas soportados en SimAS 3.0}
\label{tab:idiomas-soportados}
\end{table}

\section{Implementación}

\subsection{Clase LanguageItem}

La clase \texttt{LanguageItem} representa un idioma disponible en el sistema de internacionalización.

\textbf{Ubicación en el repositorio:} \url{https://github.com/Llamatekee/SimAS-3.0/blob/main/src/utils/LanguageItem.java}

\textbf{Características principales:}

\begin{itemize}
    \item \textbf{name}: nombre del idioma en su idioma nativo.
    \item \textbf{code}: código ISO del idioma.
    \item \textbf{flag}: bandera representativa del país/región.
    \item \textbf{locale}: objeto Locale de Java para la localización.
\end{itemize}

\textbf{Referencia:} Para una explicación más detallada de la clase LanguageItem, incluyendo su implementación completa, atributos, métodos y relaciones con otras clases, consulte el Capítulo 9 del Manual Técnico ("Diseño de clases").

\subsection{Clase LanguageListCell}

Celda personalizada para mostrar idiomas en listas desplegables del sistema de internacionalización.

\textbf{Ubicación en el repositorio:} \url{https://github.com/Llamatekee/SimAS-3.0/blob/main/src/utils/LanguageListCell.java}

\textbf{Funcionalidades:}

\begin{itemize}
    \item Muestra la bandera del país/región.
    \item Muestra el nombre del idioma.
    \item Formato visual atractivo.
    \item Integración con JavaFX.
\end{itemize}

\textbf{Referencia:} Para una explicación más detallada de la clase LanguageListCell, incluyendo su implementación completa, métodos de renderizado y integración con componentes JavaFX, consulte el Capítulo 9 del Manual Técnico ("Diseño de clases").

\subsection{Interfaz ActualizableTextos}

Interfaz fundamental que permite a los componentes actualizar sus textos dinámicamente cuando cambia el idioma.

\textbf{Ubicación:} \url{https://github.com/Llamatekee/SimAS-3.0/blob/main/src/utils/ActualizableTextos.java}

\textbf{Funcionalidades:}
\begin{itemize}
    \item Define contrato para componentes actualizables.
    \item Método \texttt{actualizarTextos(ResourceBundle)} para actualización de textos.
    \item Implementada por 25+ componentes en toda la aplicación.
    \item Soporte para actualización automática y manual.
\end{itemize}

\section{Archivos de propiedades}

\subsection{Estructura de los archivos}

Los archivos de propiedades siguen el formato estándar de Java y contienen todas las traducciones organizadas por secciones funcionales.

\textbf{Estructura de archivos:}

\begin{itemize}
    \item \textbf{Idioma base (español)}: \url{https://github.com/Llamatekee/SimAS-3.0/blob/main/src/utils/messages_es.properties}
    \item \textbf{Inglés}: \url{https://github.com/Llamatekee/SimAS-3.0/blob/main/src/utils/messages_en.properties}
    \item \textbf{Alemán}: \url{https://github.com/Llamatekee/SimAS-3.0/blob/main/src/utils/messages_de.properties}
    \item \textbf{Francés}: \url{https://github.com/Llamatekee/SimAS-3.0/blob/main/src/utils/messages_fr.properties}
    \item \textbf{Japonés}: \url{https://github.com/Llamatekee/SimAS-3.0/blob/main/src/utils/messages_ja.properties}
    \item \textbf{Portugués}: \url{https://github.com/Llamatekee/SimAS-3.0/blob/main/src/utils/messages_pt.properties}
\end{itemize}

\textbf{Organización del contenido:}

Cada archivo contiene aproximadamente 150+ claves organizadas en secciones:
\begin{itemize}
    \item \textbf{Interfaz principal}: menús, botones, títulos.
    \item \textbf{Editor de gramáticas}: pasos del asistente, validaciones.
    \item \textbf{Simulador}: estados de simulación, acciones, errores.
    \item \textbf{Mensajes del sistema}: errores, confirmaciones, notificaciones.
    \item \textbf{Ayuda contextual}: tooltips y mensajes de ayuda.
\end{itemize}

\textbf{Características técnicas:}
\begin{itemize}
    \item Codificación UTF-8 para soporte completo de caracteres internacionales.
    \item Formato clave=valor estándar de Java.
    \item Comentarios descriptivos para facilitar mantenimiento.
    \item Validación automática de integridad de traducciones.
\end{itemize}

\section{Integración con la interfaz}

\subsection{Cambio dinámico de idioma}

El sistema permite cambiar el idioma de la aplicación en tiempo de ejecución:

\begin{lstlisting}[caption=Método para cambiar idioma]
public void cambiarIdioma(LanguageItem idiomaSeleccionado) {
    currentLocale = idiomaSeleccionado.getLocale();
    bundle = ResourceBundle.getBundle("utils.messages", currentLocale);
    
    // Actualizar textos de la interfaz
    actualizarTextos();
    
    // Notificar a otros componentes
    notificarCambioIdioma();
}
\end{lstlisting}

\subsection{Actualización de textos}

Los componentes que implementan \texttt{ActualizableTextos} pueden actualizar sus textos:

\begin{lstlisting}[caption=Implementación de actualización de textos]
@Override
public void actualizarTextos(ResourceBundle bundle) {
    labelTitulo.setText(bundle.getString("menu.title"));
    labelSubtitulo.setText(bundle.getString("menu.subtitle"));
    btnEditor.setText(bundle.getString("menu.editor"));
    btnSimulador.setText(bundle.getString("menu.simulator"));
    btnAyuda.setText(bundle.getString("menu.help"));
    btnSalir.setText(bundle.getString("menu.exit"));
}
\end{lstlisting}

\section{Configuración y uso}

\subsection{Inicialización del sistema}

El sistema se inicializa al arrancar la aplicación:

\begin{lstlisting}[caption=Inicialización del sistema de i18n]
private void inicializarInternacionalizacion() {
    // Cargar idioma por defecto (español)
    currentLocale = new Locale("es");
    bundle = ResourceBundle.getBundle("utils.messages", currentLocale);
    
    // Configurar combo de idiomas
    configurarComboIdiomas();
    
    // Actualizar textos iniciales
    actualizarTextos();
}
\end{lstlisting}

\subsection{Configuración del selector de idiomas}

\begin{lstlisting}[caption=Configuración del selector de idiomas]
private void configurarComboIdiomas() {
    List<LanguageItem> idiomas = Arrays.asList(
        new LanguageItem("Espanol", "es", "espana.png"),
        new LanguageItem("English", "en", "england.png"),
        new LanguageItem("Deutsch", "de", "alemania.png"),
        new LanguageItem("Francais", "fr", "francia.png"),
        new LanguageItem("Japanese", "ja", "japon.png"),
        new LanguageItem("Portugues", "pt", "portugal.png")
    );
    
    comboIdioma.setItems(FXCollections.observableArrayList(idiomas));
    comboIdioma.setCellFactory(listView -> new LanguageListCell());
    comboIdioma.setButtonCell(new LanguageListCell());
    
    // Seleccionar idioma actual
    comboIdioma.getSelectionModel().select(
        idiomas.stream()
               .filter(idioma -> idioma.getCode().equals(currentLocale.getLanguage()))
               .findFirst()
               .orElse(idiomas.get(0))
    );
}
\end{lstlisting}

\section{Recursos gráficos}

\subsection{Banderas de países}

El sistema utiliza imágenes de banderas para representar visualmente cada idioma:

\begin{itemize}
    \item \texttt{espana.png}: bandera de España.
    \item \texttt{england.png}: bandera de Inglaterra.
    \item \texttt{alemania.png}: bandera de Alemania.
    \item \texttt{francia.png}: bandera de Francia.
    \item \texttt{japon.png}: bandera de Japón.
    \item \texttt{portugal.png}: bandera de Portugal.
\end{itemize}

\subsection{Ubicación de recursos}

Las banderas se encuentran en el directorio \texttt{src/resources/} y se cargan como recursos de la aplicación.

\section{Consideraciones técnicas}

\subsection{Rendimiento}

\begin{itemize}
    \item Los archivos de propiedades se cargan una sola vez al inicializar.
    \item El cambio de idioma es instantáneo.
    \item No hay impacto significativo en el rendimiento.
\end{itemize}

\subsection{Mantenimiento}

\begin{itemize}
    \item Fácil adición de nuevos idiomas.
    \item Estructura clara y organizada.
    \item Separación entre código y textos.
\end{itemize}

\subsection{Extensibilidad}

El sistema está diseñado para ser fácilmente extensible:

\begin{itemize}
    \item Agregar nuevos idiomas solo requiere crear un archivo de propiedades.
    \item Agregar nuevos textos requiere actualizar todos los archivos de propiedades.
    \item La interfaz se adapta automáticamente a nuevos idiomas.
\end{itemize}

\section{Mejores prácticas implementadas}

\subsection{Separación de contenido y código}

\begin{itemize}
    \item Todos los textos visibles están en archivos de propiedades.
    \item No hay cadenas hardcodeadas en el código.
    \item Fácil localización por parte de traductores.
\end{itemize}

\subsection{Gestión de recursos}

\begin{itemize}
    \item Uso eficiente de ResourceBundle.
    \item Carga lazy de recursos gráficos.
    \item Gestión adecuada de memoria.
\end{itemize}

\subsection{Experiencia de usuario}

\begin{itemize}
    \item Cambio de idioma inmediato.
    \item Interfaz intuitiva para selección de idioma.
    \item Persistencia de la preferencia de idioma.
\end{itemize}

\section{Métricas del sistema de internacionalización}

\subsection{Cobertura de internacionalización}

\begin{itemize}
    \item \textbf{6 idiomas completamente soportados}: Español, Inglés, Alemán, Francés, Japonés, Portugués.
    \item \textbf{150+ claves de traducción} por archivo de propiedades.
    \item \textbf{25+ componentes actualizables} que implementan \texttt{ActualizableTextos}.
    \item \textbf{100\% de cobertura}: todos los textos de interfaz están internacionalizados.
\end{itemize}

\subsection{Implementación técnica}

\begin{itemize}
    \item \textbf{ResourceBundle estándar}: uso de la API nativa de Java para internacionalización.
    \item \textbf{UTF-8 completo}: soporte para caracteres internacionales y símbolos especiales.
    \item \textbf{Cambio dinámico}: conmutación de idioma en tiempo de ejecución sin reinicio.
    \item \textbf{Persistencia}: mantenimiento de la selección de idioma entre sesiones.
\end{itemize}

\section{Pruebas y validación}

\subsection{Verificación de traducciones}

\begin{itemize}
    \item Validación automática de completitud de traducciones.
    \item Verificación de formato y sintaxis de archivos de propiedades.
    \item Pruebas de carga y acceso a recursos localizados.
    \item Detección automática de claves faltantes o inconsistentes.
\end{itemize}

\subsection{Pruebas de interfaz}

\begin{itemize}
    \item Verificación de adaptación automática de textos largos.
    \item Pruebas exhaustivas de cambio dinámico de idioma.
    \item Validación de recursos gráficos y banderas de países.
    \item Testing de usabilidad en diferentes idiomas.
\end{itemize}

\subsection{Herramientas de desarrollo}

\begin{itemize}
    \item Scripts de validación automática de traducciones.
    \item Herramientas de comparación entre archivos de idiomas.
    \item Sistema de logging para debugging de internacionalización.
\end{itemize}

\textbf{Referencias al código fuente:}

\begin{itemize}
    \item \textbf{Framework de i18n}: \url{https://github.com/Llamatekee/SimAS-3.0/tree/main/src/utils}
    \item \textbf{Archivos de propiedades}: \url{https://github.com/Llamatekee/SimAS-3.0/tree/main/src/utils} (\verb|messages_*.properties|)
    \item \textbf{Recursos gráficos}: \url{https://github.com/Llamatekee/SimAS-3.0/tree/main/src/resources}
\end{itemize}
