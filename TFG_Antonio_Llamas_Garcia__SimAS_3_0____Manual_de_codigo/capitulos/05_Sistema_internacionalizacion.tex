\chapter{Sistema de Internacionalización}\label{cap-internacionalizacion}

\section{Introducción}

SimAS 3.0 implementa un sistema completo de internacionalización (i18n) que permite a la aplicación adaptarse a diferentes idiomas y regiones. Este capítulo documenta la implementación, configuración y uso del sistema de internacionalización.

\section{Arquitectura del sistema}

\subsection{Componentes principales}

El sistema de internacionalización está compuesto por los siguientes elementos:

\begin{itemize}
    \item \textbf{Archivos de propiedades}: Contienen las traducciones para cada idioma
    \item \textbf{LanguageItem}: Representa un idioma disponible
    \item \textbf{LanguageListCell}: Celda personalizada para mostrar idiomas
    \item \textbf{ActualizableTextos}: Interfaz para componentes que pueden actualizar sus textos
    \item \textbf{ResourceBundle}: Mecanismo de Java para cargar recursos localizados
\end{itemize}

\subsection{Idiomas soportados}

La aplicación soporta los siguientes idiomas:

\begin{table}[H]
\centering
\begin{tabular}{|l|l|l|}
\hline
\textbf{Código} & \textbf{Idioma} & \textbf{Archivo} \\
\hline
es & Español & messages\_es.properties \\
\hline
en & English & messages\_en.properties \\
\hline
de & Deutsch & messages\_de.properties \\
\hline
fr & Français & messages\_fr.properties \\
\hline
ja & Japanese & messages\_ja.properties \\
\hline
pt & Português & messages\_pt.properties \\
\hline
\end{tabular}
\caption{Idiomas soportados en SimAS 3.0}
\label{tab:idiomas-soportados}
\end{table}

\section{Implementación}

\subsection{LanguageItem.java}

La clase \texttt{LanguageItem} representa un idioma disponible en el sistema:

\inputminted[linenos,breaklines]{java}{codigo/src/utils/LanguageItem.java}

\textbf{Características principales:}

\begin{itemize}
    \item \textbf{name}: Nombre del idioma en su idioma nativo
    \item \textbf{code}: Código ISO del idioma
    \item \textbf{flag}: Bandera representativa del país/región
    \item \textbf{locale}: Objeto Locale de Java para la localización
\end{itemize}

\subsection{LanguageListCell.java}

Celda personalizada para mostrar idiomas en listas desplegables:

\inputminted[linenos,breaklines]{java}{codigo/src/utils/LanguageListCell.java}

\textbf{Funcionalidades:}

\begin{itemize}
    \item Muestra la bandera del país/región
    \item Muestra el nombre del idioma
    \item Formato visual atractivo
    \item Integración con JavaFX
\end{itemize}

\subsection{ActualizableTextos.java}

Interfaz que permite a los componentes actualizar sus textos dinámicamente:

\inputminted[linenos,breaklines]{java}{codigo/src/utils/ActualizableTextos.java}

\section{Archivos de propiedades}

\subsection{Estructura de los archivos}

Los archivos de propiedades siguen el formato estándar de Java:

\begin{lstlisting}[language=Java, caption=Ejemplo de archivo de propiedades]
# SimAS 3.0 - Mensajes en español
# Archivo: messages_es.properties

# Menú principal
menu.title=SimAS 3.0
menu.subtitle=Simulador de Análisis Sintáctico
menu.editor=Editor de Gramáticas
menu.simulator=Simulador Descendente
menu.help=Manual de Usuario
menu.exit=Salir

# Editor
editor.title=Editor de Gramáticas
editor.step1=Términos No Terminales
editor.step2=Términos Terminales
editor.step3=Producciones
editor.step4=Tabla Predictiva

# Simulador
simulator.title=Simulador Descendente
simulator.input=Cadena de Entrada
simulator.start=Iniciar Simulación
simulator.next=Siguiente Paso
simulator.reset=Reiniciar

# Mensajes de error
error.invalid.grammar=Gramática inválida
error.conflict=Conflicto detectado en la gramática
error.syntax=Error de sintaxis
\end{lstlisting}

\subsection{Archivo en inglés}

\inputminted[linenos,breaklines,firstline=1,lastline=30]{properties}{codigo/src/utils/messages_en.properties}

\subsection{Archivo en alemán}

\inputminted[linenos,breaklines,firstline=1,lastline=30]{properties}{codigo/src/utils/messages_de.properties}

\section{Integración con la interfaz}

\subsection{Cambio dinámico de idioma}

El sistema permite cambiar el idioma de la aplicación en tiempo de ejecución:

\begin{lstlisting}[language=Java, caption=Método para cambiar idioma]
public void cambiarIdioma(LanguageItem idiomaSeleccionado) {
    currentLocale = idiomaSeleccionado.getLocale();
    bundle = ResourceBundle.getBundle("utils.messages", currentLocale);
    
    // Actualizar textos de la interfaz
    actualizarTextos();
    
    // Notificar a otros componentes
    notificarCambioIdioma();
}
\end{lstlisting}

\subsection{Actualización de textos}

Los componentes que implementan \texttt{ActualizableTextos} pueden actualizar sus textos:

\begin{lstlisting}[language=Java, caption=Implementación de actualización de textos]
@Override
public void actualizarTextos(ResourceBundle bundle) {
    labelTitulo.setText(bundle.getString("menu.title"));
    labelSubtitulo.setText(bundle.getString("menu.subtitle"));
    btnEditor.setText(bundle.getString("menu.editor"));
    btnSimulador.setText(bundle.getString("menu.simulator"));
    btnAyuda.setText(bundle.getString("menu.help"));
    btnSalir.setText(bundle.getString("menu.exit"));
}
\end{lstlisting}

\section{Configuración y uso}

\subsection{Inicialización del sistema}

El sistema se inicializa al arrancar la aplicación:

\begin{lstlisting}[language=Java, caption=Inicialización del sistema de i18n]
private void inicializarInternacionalizacion() {
    // Cargar idioma por defecto (español)
    currentLocale = new Locale("es");
    bundle = ResourceBundle.getBundle("utils.messages", currentLocale);
    
    // Configurar combo de idiomas
    configurarComboIdiomas();
    
    // Actualizar textos iniciales
    actualizarTextos();
}
\end{lstlisting}

\subsection{Configuración del selector de idiomas}

\begin{lstlisting}[language=Java, caption=Configuración del selector de idiomas]
private void configurarComboIdiomas() {
    List<LanguageItem> idiomas = Arrays.asList(
        new LanguageItem("Espanol", "es", "espana.png"),
        new LanguageItem("English", "en", "england.png"),
        new LanguageItem("Deutsch", "de", "alemania.png"),
        new LanguageItem("Francais", "fr", "francia.png"),
        new LanguageItem("Japanese", "ja", "japon.png"),
        new LanguageItem("Portugues", "pt", "portugal.png")
    );
    
    comboIdioma.setItems(FXCollections.observableArrayList(idiomas));
    comboIdioma.setCellFactory(listView -> new LanguageListCell());
    comboIdioma.setButtonCell(new LanguageListCell());
    
    // Seleccionar idioma actual
    comboIdioma.getSelectionModel().select(
        idiomas.stream()
               .filter(idioma -> idioma.getCode().equals(currentLocale.getLanguage()))
               .findFirst()
               .orElse(idiomas.get(0))
    );
}
\end{lstlisting}

\section{Recursos gráficos}

\subsection{Banderas de países}

El sistema utiliza imágenes de banderas para representar visualmente cada idioma:

\begin{itemize}
    \item \texttt{espana.png}: Bandera de España
    \item \texttt{england.png}: Bandera de Inglaterra
    \item \texttt{alemania.png}: Bandera de Alemania
    \item \texttt{francia.png}: Bandera de Francia
    \item \texttt{japon.png}: Bandera de Japón
    \item \texttt{portugal.png}: Bandera de Portugal
\end{itemize}

\subsection{Ubicación de recursos}

Las banderas se encuentran en el directorio \texttt{src/resources/} y se cargan como recursos de la aplicación.

\section{Consideraciones técnicas}

\subsection{Rendimiento}

\begin{itemize}
    \item Los archivos de propiedades se cargan una sola vez al inicializar
    \item El cambio de idioma es instantáneo
    \item No hay impacto significativo en el rendimiento
\end{itemize}

\subsection{Mantenimiento}

\begin{itemize}
    \item Fácil adición de nuevos idiomas
    \item Estructura clara y organizada
    \item Separación entre código y textos
\end{itemize}

\subsection{Extensibilidad}

El sistema está diseñado para ser fácilmente extensible:

\begin{itemize}
    \item Agregar nuevos idiomas solo requiere crear un archivo de propiedades
    \item Agregar nuevos textos requiere actualizar todos los archivos de propiedades
    \item La interfaz se adapta automáticamente a nuevos idiomas
\end{itemize}

\section{Mejores prácticas implementadas}

\subsection{Separación de contenido y código}

\begin{itemize}
    \item Todos los textos visibles están en archivos de propiedades
    \item No hay cadenas hardcodeadas en el código
    \item Fácil localización por parte de traductores
\end{itemize}

\subsection{Gestión de recursos}

\begin{itemize}
    \item Uso eficiente de ResourceBundle
    \item Carga lazy de recursos gráficos
    \item Gestión adecuada de memoria
\end{itemize}

\subsection{Experiencia de usuario}

\begin{itemize}
    \item Cambio de idioma inmediato
    \item Interfaz intuitiva para selección de idioma
    \item Persistencia de la preferencia de idioma
\end{itemize}

\section{Pruebas y validación}

\subsection{Verificación de traducciones}

\begin{itemize}
    \item Validación de completitud de traducciones
    \item Verificación de formato de archivos de propiedades
    \item Pruebas de carga de recursos
\end{itemize}

\subsection{Pruebas de interfaz}

\begin{itemize}
    \item Verificación de adaptación de textos largos
    \item Pruebas de cambio dinámico de idioma
    \item Validación de recursos gráficos
\end{itemize}
