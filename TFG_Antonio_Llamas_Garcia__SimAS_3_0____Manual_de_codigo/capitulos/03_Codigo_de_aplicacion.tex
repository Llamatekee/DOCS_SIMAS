\chapter{Código de la aplicación}\label{cap-documentacion-interna}

\section{Introducción}

Este capítulo presenta una documentación detallada del código fuente de SimAS 3.0, con referencias específicas al repositorio de GitHub donde se encuentra todo el código fuente completo. A diferencia de versiones anteriores, este capítulo no incluye fragmentos de código embebidos, sino referencias directas a los archivos en el repositorio.

\textbf{Referencias al Manual Técnico:} Para explicaciones más detalladas sobre las clases, paquetes y dependencias del sistema, incluyendo diagramas de clases completos, análisis de algoritmos, patrones de diseño implementados y métricas de complejidad, consulte:
\begin{itemize}
    \item Capítulo 8: "Diseño de paquetes" - Arquitectura completa y análisis de dependencias.
    \item Capítulo 9: "Diseño de clases" - Implementación detallada de todas las clases principales.
\end{itemize}

\section{Punto de entrada de la aplicación}

\subsection{Clase Bienvenida}

La clase \texttt{Bienvenida} es el punto de entrada de la aplicación y gestiona la pantalla de bienvenida.

\textbf{Ubicación en el repositorio:} \url{https://github.com/Llamatekee/SimAS-3.0/blob/main/src/bienvenida/Bienvenida.java}

\textbf{Funcionalidades principales:}

\begin{itemize}
    \item Punto de entrada principal de la aplicación JavaFX.
    \item Configuración de la ventana de bienvenida (sin decoraciones, dimensiones fijas).
    \item Timeline automática que transita al menú principal después de 2.5 segundos.
    \item Gestión del ciclo de vida inicial de la aplicación.
\end{itemize}

\textbf{Métodos principales:}
\begin{itemize}
    \item \texttt{start(Stage)}: inicializa la interfaz de bienvenida.
    \item \texttt{abrirMenuPrincipal()}: transita al menú principal.
    \item \texttt{main(String[])}: método main estático.
\end{itemize}

\subsection{Clase MenuPrincipal}

La clase \texttt{MenuPrincipal} es el controlador principal de navegación de la aplicación.

\textbf{Ubicación en el repositorio:} \url{https://github.com/Llamatekee/SimAS-3.0/blob/main/src/bienvenida/MenuPrincipal.java}

\textbf{Funcionalidades principales:}

\begin{itemize}
    \item Gestión completa de la interfaz principal de navegación.
    \item Sistema avanzado de pestañas para múltiples proyectos simultáneos.
    \item Soporte completo de internacionalización con cambio dinámico de idioma.
    \item Gestión de atajos de teclado para operaciones frecuentes.
    \item Coordinación entre módulos (editor, simulador, ayuda).
    \item Sistema de ventanas secundarias para gestión avanzada de pestañas.
\end{itemize}

\textbf{Métodos principales:}
\begin{itemize}
    \item \texttt{start(Stage)}: inicializa la ventana principal.
    \item \texttt{cambiarIdioma()}: gestiona el cambio dinámico de idioma.
    \item \texttt{onBtnEditorAction()}: abre el editor de gramáticas.
    \item \texttt{onBtnSimuladorAction()}: abre el simulador.
    \item \texttt{onBtnAyudaAction()}: abre el sistema de ayuda.
\end{itemize}

\section{Modelo de datos: Paquete gramatica}

\textbf{Referencia al Manual Técnico:} Para una explicación más detallada del paquete gramatica, incluyendo diagramas de clases completos, análisis de algoritmos de complejidad O(n³), patrones de diseño implementados (Composite, Factory, Strategy) y todas las relaciones de dependencia, consulte:
\begin{itemize}
    \item Capítulo 8: "Diseño de paquetes" - Arquitectura y dependencias del paquete gramatica.
    \item Capítulo 9: "Diseño de clases" - Implementación detallada de clases como Gramatica, Simbolo, Terminal, NoTerminal,
\end{itemize}

\subsection{Clase Gramatica}

La clase \texttt{Gramatica} es el núcleo del sistema de modelado de gramáticas, implementando los algoritmos fundamentales de análisis sintáctico.

\textbf{Ubicación en el repositorio:} \url{https://github.com/Llamatekee/SimAS-3.0/blob/main/src/gramatica/Gramatica.java}

\textbf{Funcionalidades principales:}

\begin{itemize}
    \item Gestión completa de símbolos terminales y no terminales.
    \item Administración de producciones gramaticales con operaciones CRUD.
    \item Implementación de algoritmos de cálculo de conjuntos PRIMERO y SIGUIENTE.
    \item Generación automática de tablas de análisis predictivo.
    \item Validación de gramáticas LL(1) con detección de conflictos.
    \item Transformaciones gramaticales (eliminación de recursividad, factorización).
    \item Persistencia y carga de gramáticas desde archivos.
    \item Integración completa con JavaFX mediante propiedades observables.
\end{itemize}

\textbf{Algoritmos implementados:}
\begin{itemize}
    \item Cálculo iterativo de conjuntos PRIMERO.
    \item Cálculo iterativo de conjuntos SIGUIENTE.
    \item Construcción de tablas predictivas O(n²).
    \item Eliminación de recursividad izquierda.
    \item Factorización automática.
\end{itemize}

\subsection{Jerarquía de símbolos}

\subsubsection{Clase Simbolo (abstracta)}

Clase base abstracta que define la interfaz común para todos los símbolos gramaticales.

\textbf{Ubicación:} \url{https://github.com/Llamatekee/SimAS-3.0/blob/main/src/gramatica/Simbolo.java}

\textbf{Responsabilidades:}
\begin{itemize}
    \item Definición de interfaz común para símbolos.
    \item Gestión de nombres y valores de símbolos.
    \item Implementación de métodos de comparación e igualdad.
    \item Soporte para propiedades JavaFX observables.  
\end{itemize}

\subsubsection{Clase Terminal}

Implementación concreta para símbolos terminales de la gramática.

\textbf{Ubicación:} \url{https://github.com/Llamatekee/SimAS-3.0/blob/main/src/gramatica/Terminal.java}

\textbf{Características específicas:}
\begin{itemize}
    \item Representación de tokens del lenguaje.
    \item Métodos específicos para análisis léxico.
    \item Comparación y validación de terminales.
    \item Integración con el analizador sintáctico.
\end{itemize}

\subsubsection{Clase NoTerminal}

Implementación concreta para símbolos no terminales de la gramática.

\textbf{Ubicación:} \url{https://github.com/Llamatekee/SimAS-3.0/blob/main/src/gramatica/NoTerminal.java}

\textbf{Características específicas:}
\begin{itemize}
    \item Gestión de conjuntos PRIMERO y SIGUIENTE.
    \item Algoritmos de cálculo iterativo.
    \item Métodos de comparación específicos.
    \item Integración con algoritmos de análisis predictivo.
\end{itemize}

\section{Algoritmos de análisis sintáctico}

\subsection{Cálculo de conjuntos PRIMERO}

El algoritmo para calcular los conjuntos PRIMERO se implementa en la clase \texttt{Gramatica}. Este algoritmo calcula, para cada símbolo no terminal, el conjunto de símbolos terminales que pueden aparecer al inicio de las cadenas derivadas de ese símbolo.

\textbf{Ubicación:} \url{https://github.com/Llamatekee/SimAS-3.0/blob/main/src/gramatica/Gramatica.java} (método \texttt{calcularFirst()})

\textbf{Algoritmo implementado:}
\begin{enumerate}
    \item Inicializar conjuntos PRIMERO vacíos para todos los no terminales.
    \item Mientras haya cambios en los conjuntos:
    \begin{enumerate}
        \item Para cada producción $A \rightarrow \alpha$
        \item Calcular PRIMERO($\alpha$)
        \item Agregar PRIMERO($\alpha$) a PRIMERO($A$)
    \end{enumerate}
    \item Repetir hasta convergencia.
\end{enumerate}

\textbf{Complejidad:} O(n³) en el peor caso, donde n es el número de símbolos.

\subsection{Cálculo de conjuntos SIGUIENTE}

El algoritmo para calcular los conjuntos SIGUIENTE determina, para cada símbolo no terminal, el conjunto de símbolos terminales que pueden aparecer inmediatamente después de ese símbolo en alguna derivación.

\textbf{Ubicación:} \url{https://github.com/Llamatekee/SimAS-3.0/blob/main/src/gramatica/Gramatica.java} (método \texttt{calcularFollow()})

\textbf{Algoritmo implementado:}
\begin{enumerate}
    \item Inicializar conjuntos SIGUIENTE vacíos.
    \item SIGUIENTE(S) = \{\$\} donde S es el símbolo inicial.
    \item Mientras haya cambios en los conjuntos:
    \begin{enumerate}
        \item Para cada producción $A \rightarrow \alpha B \beta$
        \item Agregar PRIMERO($\beta$) a SIGUIENTE($B$)
        \item Si $\beta$ puede derivar $\epsilon$, agregar SIGUIENTE($A$) a SIGUIENTE($B$)
    \end{enumerate}
    \item Repetir hasta convergencia.
\end{enumerate}

\textbf{Complejidad:} O(n³) en el peor caso.

\section{Clases del paquete simulador}

\textbf{Referencia al Manual Técnico:} Para una explicación más detallada del paquete simulador, incluyendo algoritmos de análisis sintáctico descendente predictivo, manejo de errores con funciones personalizables, patrones de diseño (Strategy, Observer, Template Method) y análisis de complejidad algorítmica, consulte:
\begin{itemize}
    \item Capítulo 8: "Diseño de paquetes" - Arquitectura completa del simulador.
    \item Capítulo 9: "Diseño de clases" - Implementación detallada de SimulacionFinal y clases relacionadas.
\end{itemize}

\subsection{Clase PanelSimulacion}

Panel básico que gestiona la interfaz simplificada del simulador.

\textbf{Ubicación en el repositorio:} \url{https://github.com/Llamatekee/SimAS-3.0/blob/main/src/simulador/PanelSimulacion.java}

\textbf{Funcionalidades principales:}

\begin{itemize}
    \item Interfaz simplificada para demostraciones de análisis sintáctico.
    \item Visualización básica de pila, entrada y resultados.
    \item Controles básicos de navegación (siguiente paso).
    \item Integración con el sistema de pestañas.
\end{itemize}

\subsection{Clase SimulacionFinal}

La clase \texttt{SimulacionFinal} es el motor principal del simulador de análisis sintáctico descendente predictivo.

\textbf{Ubicación en el repositorio:} \url{https://github.com/Llamatekee/SimAS-3.0/blob/main/src/simulador/SimulacionFinal.java}

\textbf{Características principales:}

\begin{itemize}
    \item \textbf{Algoritmo completo LL(1):} implementa el análisis sintáctico descendente predictivo con pila.
    \item \textbf{Simulación interactiva:} navegación paso a paso con avance, retroceso e ir al inicio/final.
    \item \textbf{Visualización en tiempo real:} estado de pila, entrada restante y acciones realizadas.
    \item \textbf{Generación de derivaciones:} construcción automática de la derivación izquierda.
    \item \textbf{Visualización de árboles:} generación de representaciones gráficas con Graphviz.
    \item \textbf{Manejo de errores:} sistema completo de recuperación de errores con funciones personalizables.
    \item \textbf{Historial completo:} registro detallado de todos los pasos de simulación.
    \item \textbf{Internacionalización:} soporte completo para 6 idiomas.
    \item \textbf{Informes PDF:} generación automática de informes profesionales.
\end{itemize}

\textbf{Algoritmo principal:}

El algoritmo implementado sigue estos pasos para cada avance en la simulación:

\begin{enumerate}
    \item Verificar condiciones de aceptación (pila y entrada contienen \$).
    \item Si el símbolo en cima de pila coincide con el símbolo de entrada actual:
    \begin{enumerate}
        \item Consumir símbolo de la entrada.
        \item Desapilar símbolo.
        \item Registrar acción de "emparejar".
    \end{enumerate}
    \item En caso contrario:
    \begin{enumerate}
        \item Consultar tabla predictiva.
        \item Aplicar producción o función de error correspondiente.
        \item Actualizar pila y registrar acción.
    \end{enumerate}
    \item Actualizar interfaz y historial.
\end{enumerate}

\section{Clases del paquete editor}

\subsection{Clase Editor}

La clase \texttt{Editor} es el componente principal para la creación y edición de gramáticas en SimAS 3.0.

\textbf{Ubicación en el repositorio:} \url{https://github.com/Llamatekee/SimAS-3.0/blob/main/src/editor/Editor.java}

\textbf{Características principales:}

\begin{itemize}
    \item \textbf{Interfaz completa de edición:} proporciona una interfaz completa para crear y editar gramáticas.
    \item \textbf{Validación integrada:} incluye validación automática de gramáticas con mensajes de error detallados.
    \item \textbf{Generación de informes:} crea informes PDF profesionales de las gramáticas.
    \item \textbf{Sistema de pestañas jerárquico:} gestiona relaciones padre-hijo entre pestañas.
    \item \textbf{Internacionalización:} soporte completo para múltiples idiomas.
    \item \textbf{Integración con simulador:} permite lanzar simulaciones directamente desde el editor.
    \item \textbf{Guardado y carga:} soporta guardar y cargar gramáticas desde archivos.
    \item \textbf{Estados dinámicos:} actualiza automáticamente el estado de los botones según la gramática actual.
\end{itemize}

\textbf{Constructor principal:}

Inicializa el editor con configuración completa del sistema de pestañas y relaciones jerárquicas.

\section{Clases del paquete utils}

\subsection{Clase TabManager}

La clase \texttt{TabManager} es el núcleo del sistema de gestión de pestañas de SimAS 3.0, implementando un sistema complejo de pestañas jerárquicas.

\textbf{Ubicación en el repositorio:} \url{https://github.com/Llamatekee/SimAS-3.0/blob/main/src/utils/TabManager.java}

\textbf{Características principales:}

\begin{itemize}
    \item \textbf{Gestión de pestañas jerárquica:} sistema padre-hijo para organizar pestañas relacionadas.
    \item \textbf{Agrupación de elementos:} agrupa editores y simuladores relacionados por número de grupo.
    \item \textbf{Movimiento entre ventanas:} permite arrastrar y soltar pestañas entre ventanas.
    \item \textbf{Renovación automática:} reasigna números de grupo cuando cambian las pestañas.
    \item \textbf{Internacionalización:} soporte para textos en múltiples idiomas.
    \item \textbf{Instancias múltiples:} permite múltiples instancias de editores y simuladores.
    \item \textbf{Menús contextuales:} gestiona menús contextuales personalizados para pestañas.
    \item \textbf{Persistencia de estado:} mantiene el estado de grupos y relaciones entre sesiones.
\end{itemize}

\textbf{Método principal de creación de pestañas:}

Implementa lógica compleja para determinar si usar caché o crear nueva instancia basada en las reglas jerárquicas del sistema.

\subsection{Clase SecondaryWindow}

La clase \texttt{SecondaryWindow} implementa el sistema de ventanas secundarias de SimAS 3.0.

\textbf{Ubicación en el repositorio:} \url{https://github.com/Llamatekee/SimAS-3.0/blob/main/src/utils/SecondaryWindow.java}

\textbf{Características principales:}

\begin{itemize}
    \item \textbf{Ventanas múltiples:} permite crear múltiples ventanas secundarias independientes.
    \item \textbf{Arrastre de pestañas:} soporta arrastrar y soltar pestañas entre ventanas.
    \item \textbf{Atajos de teclado:} implementa atajos específicos para ventanas secundarias.
    \item \textbf{Numeración automática:} asigna números automáticamente a las ventanas.
    \item \textbf{Internacionalización:} soporte completo para múltiples idiomas.
    \item \textbf{Integración con TabManager:} funciona perfectamente con el sistema de pestañas.
    \item \textbf{Cierre inteligente:} gestiona el cierre automático cuando no hay pestañas.
    \item \textbf{Estados persistentes:} mantiene el estado de las ventanas activas.
\end{itemize}

\subsection{Clase LanguageItem}

Representa un elemento de idioma en el sistema de internacionalización.

\textbf{Ubicación en el repositorio:} \url{https://github.com/Llamatekee/SimAS-3.0/blob/main/src/utils/LanguageItem.java}

\textbf{Funcionalidades:}
\begin{itemize}
    \item Encapsula información completa de un idioma (código, nombre, bandera).
    \item Proporciona objeto Locale para configuración regional.
    \item Implementa patrón Value Object con métodos de comparación.
    \item Integración con componentes de interfaz JavaFX.
\end{itemize}

\section{Patrones de diseño implementados}

\subsection{Patrón MVC (Modelo-Vista-Controlador)}

La aplicación implementa claramente el patrón arquitectónico Modelo-Vista-Controlador:

\begin{itemize}
    \item \textbf{Modelo}: clases en el paquete \texttt{gramatica} (Gramatica, Simbolo, Terminal, NoTerminal).
    \item \textbf{Vista}: archivos FXML en el directorio \texttt{vistas} y estilos CSS.
    \item \textbf{Controlador}: clases como Editor, SimulacionFinal, MenuPrincipal que manejan la lógica de presentación.
\end{itemize}

\textbf{Referencia:} \url{https://github.com/Llamatekee/SimAS-3.0/blob/main/src/gramatica/Gramatica.java} (modelo)

\subsection{Patrón Observer}

Utilizado extensivamente para la gestión de eventos de la interfaz de usuario:

\begin{itemize}
    \item Observadores de JavaFX para cambios en propiedades.
    \item Sistema de internacionalización reactiva.
    \item Notificación automática de cambios de estado.
\end{itemize}

\subsection{Patrón Factory Method}

Implementado para la creación especializada de componentes:

\begin{itemize}
    \item Creación de símbolos terminales y no terminales.
    \item Factorías de componentes de interfaz.
    \item Creación dinámica de pestañas y ventanas.
\end{itemize}

\textbf{Referencia:} \url{https://github.com/Llamatekee/SimAS-3.0/blob/main/src/utils/TabManager.java}

\subsection{Patrón Strategy}

Utilizado para algoritmos intercambiables:

\begin{itemize}
    \item Diferentes estrategias de análisis sintáctico.
    \item Estrategias de manejo de errores.
    \item Algoritmos de transformación gramatical.
\end{itemize}

\section{Consideraciones de rendimiento}

\subsection{Gestión de memoria}

\begin{itemize}
    \item Uso de \texttt{HashSet} para conjuntos PRIMERO y SIGUIENTE (acceso O(1)).
    \item Implementación eficiente de algoritmos de análisis sintáctico O(n³).
    \item Gestión adecuada de recursos JavaFX con limpieza automática.
    \item Cache inteligente de componentes FXML.
\end{itemize}

\subsection{Optimizaciones implementadas}

\begin{itemize}
    \item Cálculo incremental de conjuntos PRIMERO y SIGUIENTE.
    \item Cache de resultados de análisis sintáctico.
    \item Lazy loading de componentes de interfaz.
    \item Optimización de renderizado JavaFX.
\end{itemize}

\section{Tratamiento de errores}

\subsection{Validación de gramáticas}

La aplicación implementa un sistema completo de validación:

\begin{itemize}
    \item Verificación de gramáticas bien formadas.
    \item Detección automática de conflictos LL(1).
    \item Validación sintáctica de símbolos y producciones.
    \item Mensajes de error contextuales en múltiples idiomas.
\end{itemize}

\textbf{Referencia:} \url{https://github.com/Llamatekee/SimAS-3.0/blob/main/src/gramatica/Gramatica.java} (método esLL1())

\subsection{Manejo de excepciones}

Sistema robusto de manejo de errores:

\begin{itemize}
    \item Excepciones específicas para diferentes tipos de errores.
    \item Mensajes de error informativos para el usuario.
    \item Recuperación graceful con estrategias de fallback.
    \item Logging detallado para debugging.
\end{itemize}

\subsection{Funciones de error personalizables}

\begin{itemize}
    \item Sistema de 7 funciones de error predefinidas.
    \item Creación de funciones de error personalizadas.    
    \item Recuperación automática de errores durante simulación.
    \item Configuración de estrategias de manejo de errores.
\end{itemize}

