\chapter{Compilación y Despliegue}\label{cap-compilacion-despliegue}

\section{Introducción}

Este capítulo describe el proceso completo de compilación, empaquetado y despliegue de la aplicación SimAS 3.0. Se incluyen los scripts de construcción, configuración de dependencias y procedimientos para generar ejecutables nativos.

\section{Herramientas de construcción}

\subsection{Java Development Kit (JDK)}

\textbf{Requisitos:}
\begin{itemize}
    \item JDK 17 o superior
    \item Herramienta \texttt{jpackage} (incluida desde JDK 14+)
    \item Variables de entorno configuradas correctamente
\end{itemize}

\textbf{Verificación de instalación:}
\begin{lstlisting}[language=bash, caption=Verificación del JDK]
java -version
javac -version
jpackage --version
\end{lstlisting}

\subsection{JavaFX SDK}

\textbf{Versión utilizada:} JavaFX 17.0.12

\textbf{Ubicación:} \texttt{lib/javafx-sdk-17.0.12/}

\textbf{Componentes incluidos:}
\begin{itemize}
    \item \texttt{lib/}: Librerías JavaFX
    \item \texttt{bin/}: Herramientas de JavaFX
    \item \texttt{legal/}: Licencias y documentación legal
\end{itemize}

\section{Scripts de construcción}

\subsection{build.sh (Unix/Linux/macOS)}

Script principal para sistemas Unix-like:

\inputminted[linenos,breaklines]{bash}{codigo/build.sh}

\textbf{Funcionalidades:}

\begin{itemize}
    \item \textbf{Líneas 1-10}: Configuración de variables y verificación de dependencias
    \item \textbf{Líneas 12-15}: Creación de directorios necesarios
    \item \textbf{Líneas 17-25}: Compilación del código fuente Java
    \item \textbf{Líneas 27-35}: Copia de recursos y dependencias
    \item \textbf{Líneas 37-45}: Creación del archivo JAR ejecutable
    \item \textbf{Líneas 47-50}: Generación del manifest con información de la aplicación
\end{itemize}

\subsection{build.bat (Windows)}

Script equivalente para sistemas Windows:

\inputminted[linenos,breaklines]{batch}{codigo/build.bat}

\textbf{Diferencias con build.sh:}
\begin{itemize}
    \item Sintaxis de comandos de Windows
    \item Variables de entorno específicas de Windows
    \item Rutas con separadores de Windows
\end{itemize}

\subsection{create-standalone-app.sh}

Script para crear aplicación independiente en macOS:

\inputminted[linenos,breaklines]{bash}{codigo/create-standalone-app.sh}

\textbf{Características principales:}

\begin{itemize}
    \item Utiliza \texttt{jpackage} para crear aplicación nativa
    \item Incluye todas las dependencias JavaFX
    \item Genera archivo \texttt{.app} ejecutable
    \item Configuración específica para macOS
\end{itemize}

\section{Proceso de compilación}

\subsection{Paso 1: Preparación del entorno}

\begin{enumerate}
    \item Verificar instalación de JDK 17+
    \item Configurar variables de entorno:
    \begin{lstlisting}[language=bash]
    export JAVA_HOME=/path/to/jdk17
    export PATH=$JAVA_HOME/bin:$PATH
    \end{lstlisting}
    \item Verificar acceso a JavaFX SDK
    \item Dar permisos de ejecución a scripts:
    \begin{lstlisting}[language=bash]
    chmod +x build.sh
    chmod +x create-standalone-app.sh
    \end{lstlisting}
\end{enumerate}

\subsection{Paso 2: Compilación del código fuente}

\begin{lstlisting}[language=bash, caption=Compilación del proyecto]
# Para sistemas Unix/Linux/macOS
./build.sh

# Para sistemas Windows
build.bat
\end{lstlisting}

\textbf{Proceso interno:}

\begin{enumerate}
    \item Creación de directorio \texttt{build/}
    \item Compilación de archivos Java a bytecode
    \item Copia de recursos (imágenes, FXML, propiedades)
    \item Copia de librerías JavaFX
    \item Generación del archivo JAR
\end{enumerate}

\subsection{Paso 3: Estructura de archivos generada}

Después de la compilación, se genera la siguiente estructura:

\begin{lstlisting}[language=bash, caption=Estructura del directorio build]
build/
|-- SimAS.jar                    # Archivo JAR principal
|-- lib/                         # Librerías JavaFX
|   |-- javafx.controls.jar
|   |-- javafx.fxml.jar
|   |-- ...
|-- resources/                   # Recursos de la aplicación
|   |-- logo.png
|   |-- icons/
|   |-- ...
|-- vistas/                      # Archivos FXML
|   |-- MenuPrincipal.fxml
|   |-- Bienvenida.fxml
|   |-- ...
|-- utils/                       # Archivos de propiedades
|   |-- messages_es.properties
|   |-- messages_en.properties
|   |-- ...
|-- MANIFEST.MF                  # Manifest del JAR
\end{lstlisting}

\section{Creación de ejecutables nativos}

\subsection{Usando jpackage}

\textbf{Comando básico:}
\begin{lstlisting}[language=bash, caption=Creación de ejecutable con jpackage]
jpackage --input build \
         --main-jar SimAS.jar \
         --main-class bienvenida.Bienvenida \
         --name SimAS \
         --app-version 3.0 \
         --vendor "Antonio Llamas García" \
         --description "Simulador de Análisis Sintáctico" \
         --dest dist
\end{lstlisting}

\textbf{Parámetros específicos por plataforma:}

\subsubsection{macOS}
\begin{lstlisting}[language=bash, caption=Configuración para macOS]
jpackage --input build \
         --main-jar SimAS.jar \
         --main-class bienvenida.Bienvenida \
         --name SimAS \
         --type dmg \
         --app-version 3.0 \
         --mac-package-identifier com.simas.app \
         --mac-package-name "SimAS 3.0" \
         --dest dist
\end{lstlisting}

\subsubsection{Windows}
\begin{lstlisting}[language=bash, caption=Configuración para Windows]
jpackage --input build \
         --main-jar SimAS.jar \
         --main-class bienvenida.Bienvenida \
         --name SimAS \
         --type msi \
         --app-version 3.0 \
         --win-dir-chooser \
         --win-menu \
         --win-shortcut \
         --dest dist
\end{lstlisting}

\subsubsection{Linux}
\begin{lstlisting}[language=bash, caption=Configuración para Linux]
jpackage --input build \
         --main-jar SimAS.jar \
         --main-class bienvenida.Bienvenida \
         --name SimAS \
         --type deb \
         --app-version 3.0 \
         --linux-shortcut \
         --dest dist
\end{lstlisting}

\subsection{Aplicación independiente para macOS}

El script \texttt{create-standalone-app.sh} crea una aplicación completamente independiente:

\begin{lstlisting}[language=bash, caption=Creación de aplicación independiente]
./create-standalone-app.sh
\end{lstlisting}

\textbf{Resultado:}
\begin{itemize}
    \item Archivo \texttt{SimAS.app} en \texttt{dist-standalone/}
    \item Aplicación completamente autónoma
    \item No requiere Java instalado en el sistema
    \item Incluye todas las dependencias necesarias
\end{itemize}

\section{Configuración del manifest}

\subsection{MANIFEST.MF}

El archivo manifest contiene metadatos importantes:

\inputminted[linenos,breaklines]{properties}{codigo/build/MANIFEST.MF}

\textbf{Atributos principales:}

\begin{itemize}
    \item \texttt{Main-Class}: Clase principal de la aplicación
    \item \texttt{Class-Path}: Ruta a las librerías JavaFX
    \item \texttt{Implementation-Title}: Nombre de la aplicación
    \item \texttt{Implementation-Version}: Versión de la aplicación
    \item \texttt{Implementation-Vendor}: Desarrollador
\end{itemize}

\section{Distribución}

\subsection{Requisitos para usuarios finales}

\textbf{Opción 1: JAR ejecutable}
\begin{itemize}
    \item Java Runtime Environment (JRE) 17+
    \item JavaFX Runtime (incluido en JRE 11+ o por separado)
\end{itemize}

\textbf{Opción 2: Aplicación nativa}
\begin{itemize}
    \item No requiere Java instalado
    \item Ejecutable nativo del sistema operativo
    \item Mayor tamaño de descarga
\end{itemize}

\subsection{Instrucciones de instalación}

\subsubsection{Para desarrolladores}
\begin{enumerate}
    \item Clonar el repositorio
    \item Ejecutar \texttt{./build.sh} o \texttt{build.bat}
    \item Ejecutar \texttt{java -jar build/SimAS.jar}
\end{enumerate}

\subsubsection{Para usuarios finales}
\begin{enumerate}
    \item Descargar la aplicación desde el repositorio
    \item Ejecutar \texttt{./dist-standalone/SimAS.app} (macOS)
    \item O ejecutar \texttt{java -jar SimAS.jar} (requiere Java)
\end{enumerate}

\section{Solución de problemas}

\subsection{Errores comunes}

\subsubsection{Error: \texttt{JavaFX runtime components are missing}}
\textbf{Causa:} JavaFX no está disponible en el classpath
\textbf{Solución:} Asegurar que las librerías JavaFX estén en \texttt{lib/}

\subsubsection{Error: \texttt{Main class not found}}
\textbf{Causa:} La clase principal no está especificada correctamente
\textbf{Solución:} Verificar el manifest y la estructura de paquetes

\subsubsection{Error: \texttt{Permission denied}}
\textbf{Causa:} Scripts sin permisos de ejecución
\textbf{Solución:} \texttt{chmod +x build.sh create-standalone-app.sh}

\subsection{Verificación de la instalación}

\begin{lstlisting}[language=bash, caption=Verificación de la instalación]
# Verificar que la aplicación se ejecuta
java -jar build/SimAS.jar

# Verificar estructura de archivos
ls -la build/

# Verificar manifest
unzip -p build/SimAS.jar META-INF/MANIFEST.MF
\end{lstlisting}

\section{Optimizaciones de rendimiento}

\subsection{Configuración de JVM}

\textbf{Parámetros recomendados:}
\begin{lstlisting}[language=bash, caption=Parámetros JVM optimizados]
java -Xmx512m -Xms256m -jar SimAS.jar
\end{lstlisting}

\subsection{Reducción del tamaño}

\begin{itemize}
    \item Uso de ProGuard para ofuscación y minificación
    \item Eliminación de dependencias no utilizadas
    \item Compresión de recursos
\end{itemize}

\section{Automatización con CI/CD}

\subsection{GitHub Actions}

Ejemplo de workflow para compilación automática:

\begin{lstlisting}[language=bash, caption=Workflow de GitHub Actions]
name: Build SimAS
on: [push, pull_request]

jobs:
  build:
    runs-on: ${{ matrix.os }}
    strategy:
      matrix:
        os: [ubuntu-latest, windows-latest, macos-latest]
    
    steps:
    - uses: actions/checkout@v2
    
    - name: Set up JDK 17
      uses: actions/setup-java@v2
      with:
        java-version: '17'
        distribution: 'adopt'
    
    - name: Build application
      run: |
        chmod +x build.sh
        ./build.sh
    
    - name: Create native executable
      run: |
        chmod +x create-standalone-app.sh
        ./create-standalone-app.sh
\end{lstlisting}

\section{Consideraciones de seguridad}

\subsection{Firma de código}

Para distribuir la aplicación de forma segura:

\begin{itemize}
    \item Firmar el JAR con certificado digital
    \item Firmar ejecutables nativos
    \item Verificar integridad de archivos
\end{itemize}

\subsection{Verificación de dependencias}

\begin{itemize}
    \item Verificar checksums de librerías JavaFX
    \item Actualizar dependencias regularmente
    \item Usar versiones estables y soportadas
\end{itemize}
