\chapter{Compilación y Despliegue}\label{cap-compilacion-despliegue}

\section{Introducción}

Este capítulo describe el proceso completo de compilación, empaquetado y despliegue de la aplicación SimAS 3.0. Se incluyen referencias a los scripts de construcción en el repositorio, configuración de dependencias y procedimientos para generar ejecutables nativos multiplataforma.

\textbf{Referencias al Manual Técnico:} Para información detallada sobre la arquitectura del sistema de construcción, configuración de entornos de desarrollo y análisis de rendimiento del sistema, consulte:
\begin{itemize}
    \item Capítulo 8: "Diseño de paquetes" - Estructura del sistema y dependencias.
    \item Capítulo 9: "Diseño de clases" - Clases relacionadas con configuración y utilidades del sistema.
\end{itemize}

\textbf{Ubicación en el repositorio:} \url{https://github.com/Llamatekee/SimAS-3.0/tree/main}

\section{Herramientas de construcción}

\subsection{Java Development Kit (JDK)}

\textbf{Requisitos del sistema:}
\begin{itemize}
    \item JDK 17 o superior (OpenJDK o Oracle JDK).
    \item Herramienta \texttt{jpackage} (incluida desde JDK 14+).
    \item Variables de entorno \verb|JAVA_HOME| y \verb|PATH| configuradas.
    \item JavaFX SDK 17.0.12 (no incluido en JDK estándar).
\end{itemize}

\textbf{Verificación de instalación:}
\begin{itemize}
    \item \texttt{java -version} - Verificar versión de Java.
    \item \texttt{javac -version} - Verificar compilador Java.
    \item \texttt{jpackage --version} - Verificar herramienta de empaquetado.
\end{itemize}

\textbf{Configuración recomendada:}
\begin{lstlisting}[caption=Configuración de variables de entorno]
export JAVA_HOME=/path/to/jdk17
export PATH=$JAVA_HOME/bin:$PATH
export JAVAFX_HOME=/path/to/javafx-sdk-17.0.12
\end{lstlisting}

\subsection{JavaFX SDK}

\textbf{Versión utilizada:} JavaFX 17.0.12

\textbf{Ubicación:} \texttt{lib/javafx-sdk-17.0.12/}

\textbf{Componentes incluidos:}
\begin{itemize}
    \item \texttt{lib/}: Librerías JavaFX.
    \item \texttt{bin/}: Herramientas de JavaFX.
    \item \texttt{legal/}: Licencias y documentación legal.
\end{itemize}

\section{Scripts de construcción}

\subsection{Script build.sh (Unix/Linux/macOS)}

Script principal para compilación en sistemas Unix-like.

\textbf{Ubicación:} \url{https://github.com/Llamatekee/SimAS-3.0/blob/main/build.sh}

\textbf{Funcionalidades principales:}

\begin{itemize}
    \item Configuración automática de variables de entorno.
    \item Verificación de dependencias (JDK, JavaFX).
    \item Creación de estructura de directorios \texttt{build/}.
    \item Compilación completa del código fuente Java.
    \item Copia de recursos (FXML, CSS, imágenes, internacionalización).
    \item Copia de librerías JavaFX al directorio de construcción.  
    \item Generación del archivo JAR ejecutable con manifest.
    \item Optimización y minificación del paquete final.
\end{itemize}

\subsection{Script build.bat (Windows)}

Script equivalente para compilación en sistemas Windows.

\textbf{Ubicación:} \url{https://github.com/Llamatekee/SimAS-3.0/blob/main/build.bat}

\textbf{Características específicas:}
\begin{itemize}
    \item Sintaxis de comandos nativa de Windows (\texttt{cmd}).
    \item Manejo de variables de entorno de Windows.
    \item Rutas con separadores de directorio de Windows (\textbackslash).
    \item Integración con herramientas de desarrollo Windows.
    \item Soporte para PowerShell y Command Prompt.
\end{itemize}

\subsection{Script create-standalone-app.sh}

Script especializado para crear aplicaciones independientes.

\textbf{Ubicación:} \url{https://github.com/Llamatekee/SimAS-3.0/blob/main/create-standalone-app.sh}

\textbf{Características principales:}

\begin{itemize}
    \item Utiliza \texttt{jpackage} para crear aplicaciones nativas.
    \item Incluye JRE embebido (no requiere Java instalado).
    \item Empaqueta todas las dependencias JavaFX.
    \item Genera ejecutables nativos (.app, .exe, .deb).
    \item Configuración específica por plataforma.
    \item Optimización de tamaño y rendimiento.
\end{itemize}

\textbf{Resultado:} Aplicación completamente autónoma en \texttt{dist-standalone/}

\section{Proceso de compilación}

\subsection{Paso 1: Preparación del entorno}

\begin{enumerate}
    \item Clonar el repositorio:
    \begin{lstlisting}[]
    git clone https://github.com/Llamatekee/SimAS-3.0.git
    cd SimAS-3.0
    \end{lstlisting}

    \item Verificar instalación de JDK 17+ y JavaFX SDK
    \item Configurar variables de entorno (si no están configuradas globalmente)
    \item Dar permisos de ejecución a scripts:
    \begin{lstlisting}[]
    chmod +x build.sh
    chmod +x create-standalone-app.sh
    \end{lstlisting}
\end{enumerate}

\subsection{Paso 2: Compilación del código fuente}

\begin{itemize}
    \item \textbf{Sistemas Unix/Linux/macOS:}
    \begin{lstlisting}[]
    ./build.sh
    \end{lstlisting}

    \item \textbf{Sistemas Windows:}
    \begin{lstlisting}
    build.bat
    \end{lstlisting}
\end{itemize}

\textbf{Proceso interno:} El script automatiza la compilación completa incluyendo verificación de dependencias, creación de directorios, compilación Java, copia de recursos y generación del JAR.

\subsection{Paso 3: Estructura de archivos generada}

Después de la compilación, se genera la siguiente estructura:

\begin{lstlisting}[, caption=Estructura del directorio build]
build/
|-- SimAS.jar                    # Archivo JAR principal
|-- lib/                         # Librerías JavaFX
|   |-- javafx.controls.jar
|   |-- javafx.fxml.jar
|   |-- ...
|-- resources/                   # Recursos de la aplicación
|   |-- logo.png
|   |-- icons/
|   |-- ...
|-- vistas/                      # Archivos FXML
|   |-- MenuPrincipal.fxml
|   |-- Bienvenida.fxml
|   |-- ...
|-- utils/                       # Archivos de propiedades
|   |-- messages_es.properties
|   |-- messages_en.properties
|   |-- ...
|-- MANIFEST.MF                  # Manifest del JAR
\end{lstlisting}

\section{Creación de ejecutables nativos}

\subsection{Usando jpackage}

\textbf{Comando básico:}
\begin{lstlisting}[, caption=Creación de ejecutable con jpackage]
jpackage --input build \
         --main-jar SimAS.jar \
         --main-class bienvenida.Bienvenida \
         --name SimAS \
         --app-version 3.0 \
         --vendor "Antonio Llamas García" \
         --description "Simulador de Análisis Sintáctico" \
         --dest dist
\end{lstlisting}

\textbf{Parámetros específicos por plataforma:}

\subsubsection{macOS}
\begin{lstlisting}[, caption=Configuración para macOS]
jpackage --input build \
         --main-jar SimAS.jar \
         --main-class bienvenida.Bienvenida \
         --name SimAS \
         --type dmg \
         --app-version 3.0 \
         --mac-package-identifier com.simas.app \
         --mac-package-name "SimAS 3.0" \
         --dest dist
\end{lstlisting}

\subsubsection{Windows}
\begin{lstlisting}[, caption=Configuración para Windows]
jpackage --input build \
         --main-jar SimAS.jar \
         --main-class bienvenida.Bienvenida \
         --name SimAS \
         --type msi \
         --app-version 3.0 \
         --win-dir-chooser \
         --win-menu \
         --win-shortcut \
         --dest dist
\end{lstlisting}

\subsubsection{Linux}
\begin{lstlisting}[, caption=Configuración para Linux]
jpackage --input build \
         --main-jar SimAS.jar \
         --main-class bienvenida.Bienvenida \
         --name SimAS \
         --type deb \
         --app-version 3.0 \
         --linux-shortcut \
         --dest dist
\end{lstlisting}

\subsection{Aplicación independiente para macOS}

El script \texttt{create-standalone-app.sh} crea una aplicación completamente independiente:

\begin{lstlisting}[, caption=Creación de aplicación independiente]
./create-standalone-app.sh
\end{lstlisting}

\textbf{Resultado:}
\begin{itemize}
    \item Archivo \texttt{SimAS.app} en \texttt{dist-standalone/}.
    \item Aplicación completamente autónoma.
    \item No requiere Java instalado en el sistema.
    \item Incluye todas las dependencias necesarias.
\end{itemize}

\section{Configuración del manifest}

\subsection{Archivo MANIFEST.MF}

El archivo manifest contiene metadatos importantes sobre la aplicación.

\textbf{Ubicación:} \url{https://github.com/Llamatekee/SimAS-3.0/blob/main/build/MANIFEST.MF}

\textbf{Atributos principales:}

\begin{itemize}
    \item \texttt{Main-Class}: \texttt{bienvenida.Bienvenida} - Clase principal de la aplicación.
    \item \texttt{Class-Path}: rutas a las librerías JavaFX embebidas.
    \item \texttt{Implementation-Title}: nombre completo de la aplicación.
    \item \texttt{Implementation-Version}: versión actual (3.0).
    \item \texttt{Implementation-Vendor}: información del desarrollador.
    \item \texttt{Specification-Version}: versión de especificación.
\end{itemize}

\textbf{Funciones del manifest:}
\begin{itemize}
    \item Define el punto de entrada de la aplicación.
    \item Especifica dependencias de classpath.
    \item Proporciona metadatos para herramientas de desarrollo.
    \item Habilita ejecución con \texttt{java -jar SimAS.jar}.
\end{itemize}

\section{Distribución}

\subsection{Requisitos para usuarios finales}

\textbf{Opción 1: JAR ejecutable}
\begin{itemize}
    \item Java Runtime Environment (JRE) 17+.
    \item JavaFX Runtime (incluido en JRE 11+ o por separado).
\end{itemize}

\textbf{Opción 2: Aplicación nativa}
\begin{itemize}
    \item No requiere Java instalado.
    \item Ejecutable nativo del sistema operativo.
    \item Mayor tamaño de descarga.
\end{itemize}

\subsection{Instrucciones de instalación}

\subsubsection{Para desarrolladores}
\begin{enumerate}
    \item Clonar el repositorio.
    \item Ejecutar \texttt{./build.sh} o \texttt{build.bat}.
    \item Ejecutar \texttt{java -jar build/SimAS.jar}.
\end{enumerate}

\subsubsection{Para usuarios finales}
\begin{enumerate}
    \item Descargar la aplicación desde el repositorio.
    \item Ejecutar \texttt{./dist-standalone/SimAS.app} (macOS).
    \item O ejecutar \texttt{java -jar SimAS.jar} (requiere Java).
\end{enumerate}

\section{Solución de problemas}

\subsection{Errores comunes}

\subsubsection{Error: \texttt{JavaFX runtime components are missing}}
\textbf{Causa:} JavaFX no está disponible en el classpath.
\textbf{Solución:} asegurar que las librerías JavaFX estén en \texttt{lib/}.

\subsubsection{Error: \texttt{Main class not found}}
\textbf{Causa:} la clase principal no está especificada correctamente.
\textbf{Solución:} verificar el manifest y la estructura de paquetes.

\subsubsection{Error: \texttt{Permission denied}}
\textbf{Causa:} scripts sin permisos de ejecución.
\textbf{Solución:} \texttt{chmod +x build.sh create-standalone-app.sh}.

\subsection{Verificación de la instalación}

\begin{itemize}
    \item \textbf{Ejecutar la aplicación:}
    \begin{lstlisting}[]
    java -jar build/SimAS.jar
    \end{lstlisting}

    \item \textbf{Verificar estructura de archivos:}
    \begin{lstlisting}[]
    ls -la build/
    \end{lstlisting}

    \item \textbf{Verificar manifest:}
    \begin{lstlisting}[]
    unzip -p build/SimAS.jar META-INF/MANIFEST.MF
    \end{lstlisting}
\end{itemize}

\section{Optimizaciones de rendimiento}

\subsection{Configuración de JVM}

\textbf{Parámetros recomendados para ejecución:}
\begin{lstlisting}[, caption=Parámetros JVM optimizados]
java -Xmx512m -Xms256m -jar SimAS.jar
\end{lstlisting}

\textbf{Explicación de parámetros:}
\begin{itemize}
    \item \texttt{-Xmx512m}: máximo heap de 512MB.
    \item \texttt{-Xms256m}: heap inicial de 256MB.
    \item \texttt{-jar}: ejecutar archivo JAR.
\end{itemize}

\subsection{Reducción del tamaño}

\begin{itemize}
    \item Uso de ProGuard para ofuscación y minificación del código.
    \item Eliminación de dependencias no utilizadas con herramientas de análisis.
    \item Compresión de recursos (imágenes, archivos de configuración).
    \item Optimización de archivos JAR con \texttt{pack200}.
\end{itemize}

\section{Automatización con CI/CD}

\subsection{GitHub Actions}

El repositorio incluye configuración para compilación automática mediante GitHub Actions.

\textbf{Ubicación del workflow:} \url{https://github.com/Llamatekee/SimAS-3.0/tree/main/.github/workflows}

\textbf{Características del pipeline CI/CD:}

\begin{itemize}
    \item \textbf{Compilación multiplataforma:} Linux, macOS y Windows.
    \item \textbf{Generación de releases:} generación automática de ejecutables.
    \item \textbf{Despliegue continuo:} publicación automática de versiones.
\end{itemize}

\subsection{Beneficios de la automatización}

\begin{itemize}
    \item Verificación automática de cambios en cada commit.
    \item Compilación consistente en múltiples plataformas.
    \item Detección temprana de errores de compilación.
    \item Distribución simplificada de versiones.
    \item Historial completo de construcción.
\end{itemize}

\section{Referencias al código fuente}

\begin{itemize}
    \item \textbf{Scripts de construcción:} \url{https://github.com/Llamatekee/SimAS-3.0/tree/main}
    \begin{itemize}
        \item \texttt{build.sh} - Compilación Unix/Linux/macOS.
        \item \texttt{build.bat} - Compilación Windows.
        \item \texttt{create-standalone-app.sh} - Aplicación independiente.
    \end{itemize}
    \item \textbf{Configuración de dependencias:} \url{https://github.com/Llamatekee/SimAS-3.0/tree/main/lib}
    \item \textbf{Ejecutables generados:} \url{https://github.com/Llamatekee/SimAS-3.0/tree/main/dist-standalone}
    \item \textbf{Documentación CI/CD:} \url{https://github.com/Llamatekee/SimAS-3.0/tree/main/.github}
\end{itemize}

\section{Conclusión}

El sistema de compilación y despliegue de SimAS 3.0 proporciona una solución completa y multiplataforma para el desarrollo y distribución de la aplicación. Los scripts automatizados facilitan el proceso de construcción, mientras que el uso de jpackage permite generar ejecutables nativos independientes.

\textbf{Ventajas del sistema implementado:}

\begin{itemize}
    \item \textbf{Multiplataforma:} soporte completo para Windows, macOS y Linux.
    \item \textbf{Automatización:} scripts que automatizan todo el proceso de construcción.
    \item \textbf{Independencia:} aplicaciones que no requieren Java instalado.
    \item \textbf{Calidad:} verificación automática de dependencias y configuración.
    \item \textbf{Mantenibilidad:} código bien estructurado y documentado.
\end{itemize}

\section{Consideraciones de seguridad}

\subsection{Firma de código}

Para distribuir la aplicación de forma segura:

\begin{itemize}
    \item Firmar el JAR con certificado digital.
    \item Firmar ejecutables nativos.
    \item Verificar integridad de archivos.
\end{itemize}

\subsection{Verificación de dependencias}

\begin{itemize}
    \item Verificar checksums de librerías JavaFX.
    \item Actualizar dependencias regularmente.
    \item Usar versiones estables y soportadas.
\end{itemize}
