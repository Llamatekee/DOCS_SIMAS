\chapter{Recursos y Requisitos de desarrollo y ejecución}\label{cap-requisitos-desarrollo-ejecucion}

\section{Introducción}

Este capítulo describe los recursos necesarios para el desarrollo, compilación y ejecución de la aplicación SimAS 3.0. Se incluyen tanto los requisitos técnicos como las herramientas de desarrollo utilizadas en el proyecto.

\section{Requisitos del sistema}

\subsection{Requisitos mínimos de hardware}

Para la ejecución de SimAS 3.0 se requieren los siguientes recursos mínimos de hardware:

\begin{itemize}
    \item \textbf{Procesador}: Intel Core i3 o equivalente AMD
    \item \textbf{Memoria RAM}: 4 GB (recomendado 8 GB)
    \item \textbf{Espacio en disco}: 500 MB para la aplicación y dependencias
    \item \textbf{Resolución de pantalla}: 1024x768 píxeles (recomendado 1366x768 o superior)
    \item \textbf{Conectividad}: No requerida para funcionamiento básico
\end{itemize}

\subsection{Requisitos de software}

\subsubsection{Java Runtime Environment (JRE)}
\begin{itemize}
    \item \textbf{Versión requerida}: Java 17 o superior
    \item \textbf{Distribución}: Oracle JDK, OpenJDK, o cualquier distribución compatible
    \item \textbf{Arquitectura}: Compatible con x64, ARM64 (Apple Silicon)
\end{itemize}

\subsubsection{Sistemas operativos soportados}
\begin{itemize}
    \item \textbf{Windows}: Windows 10 o superior (x64)
    \item \textbf{macOS}: macOS 10.15 (Catalina) o superior
    \item \textbf{Linux}: Distribuciones modernas con soporte para JavaFX
\end{itemize}

\section{Entorno de desarrollo}

\subsection{Java Development Kit (JDK)}
\begin{itemize}
    \item \textbf{Versión}: JDK 17 o superior
    \item \textbf{Recomendado}: OpenJDK 17 LTS
    \item \textbf{Configuración}: Variables de entorno JAVA\_HOME configuradas correctamente
\end{itemize}

\subsection{JavaFX SDK}
\begin{itemize}
    \item \textbf{Versión}: JavaFX 17.0.12
    \item \textbf{Ubicación}: Incluido en el directorio \texttt{lib/javafx-sdk-17.0.12/}
    \item \textbf{Propósito}: Proporciona las librerías necesarias para la interfaz gráfica
\end{itemize}

\subsection{Herramientas de construcción}
\begin{itemize}
    \item \textbf{jpackage}: Incluido en JDK 14+ para crear ejecutables nativos
    \item \textbf{Scripts de construcción}: 
    \begin{itemize}
        \item \texttt{build.sh} - Para sistemas Unix/Linux/macOS
        \item \texttt{build.bat} - Para sistemas Windows
        \item \texttt{create-standalone-app.sh} - Para crear aplicación independiente
    \end{itemize}
\end{itemize}

\section{Arquitectura del sistema}

\subsection{Visión general}

SimAS 3.0 sigue una arquitectura de capas bien definida que separa las responsabilidades de cada componente:

\begin{figure}[H]
\centering
\begin{tabular}{|c|}
\hline
\textbf{Capa de Presentación} \\
\textit{JavaFX + FXML} \\
\hline
$\downarrow$ \\
\hline
\textbf{Capa de Control} \\
\textit{Controladores} \\
\hline
$\downarrow$ \\
\hline
\textbf{Lógica de Negocio} \\
\textit{Modelos + Algoritmos} \\
\hline
$\downarrow$ \\
\hline
\textbf{Capa de Datos} \\
\textit{Gramáticas + Tablas} \\
\hline
\end{tabular}
\caption{Arquitectura en capas de SimAS 3.0}
\label{fig:arquitectura-capas}
\end{figure}

\subsection{Patrones de diseño utilizados}

\subsubsection{Modelo-Vista-Controlador (MVC)}
\begin{itemize}
    \item \textbf{Modelo}: Clases en el paquete \texttt{gramatica} que representan las entidades del dominio
    \item \textbf{Vista}: Archivos FXML en el directorio \texttt{vistas}
    \item \textbf{Controlador}: Clases Java que manejan la lógica de presentación
\end{itemize}

\subsubsection{Observer Pattern}
\begin{itemize}
    \item Utilizado para la gestión de eventos de la interfaz de usuario
    \item Implementado a través de los mecanismos de JavaFX
\end{itemize}

\subsubsection{Factory Pattern}
\begin{itemize}
    \item Para la creación de diferentes tipos de símbolos (terminales, no terminales)
    \item En la generación de componentes de la interfaz de usuario
\end{itemize}

\section{Organización modular}

\subsection{Estructura de paquetes}

La aplicación está organizada en los siguientes paquetes principales:

\begin{description}
    \item[\texttt{bienvenida}] Gestión de la pantalla de bienvenida y menú principal
    \item[\texttt{editor}] Editor de gramáticas con múltiples paneles de configuración
    \item[\texttt{gramatica}] Modelo de datos y algoritmos para gramáticas libres de contexto
    \item[\texttt{simulador}] Simulador de análisis sintáctico descendente predictivo
    \item[\texttt{utils}] Utilidades generales, internacionalización y gestión de ventanas
    \item[\texttt{centroayuda}] Sistema de ayuda y documentación integrada
    \item[\texttt{vistas}] Archivos FXML que definen las interfaces de usuario
\end{description}

\subsection{Dependencias entre módulos}

\begin{figure}[H]
\centering
\begin{tabular}{|l|l|}
\hline
\textbf{Módulo} & \textbf{Dependencias} \\
\hline
bienvenida & editor, simulador, utils, gramatica \\
\hline
editor & gramatica, utils, vistas \\
\hline
simulador & gramatica, utils, vistas \\
\hline
gramatica & (independiente) \\
\hline
utils & (independiente) \\
\hline
centroayuda & utils \\
\hline
vistas & utils \\
\hline
\end{tabular}
\caption{Dependencias entre módulos de SimAS 3.0}
\label{fig:dependencias-modulos}
\end{figure}

\section{Recursos de compilación}

\subsection{Scripts de construcción}

\subsubsection{build.sh (Unix/Linux/macOS)}
\begin{itemize}
    \item Compila el código fuente Java
    \item Empaqueta las dependencias JavaFX
    \item Genera el archivo JAR ejecutable
    \item Crea la estructura de directorios necesaria
\end{itemize}

\subsubsection{build.bat (Windows)}
\begin{itemize}
    \item Equivalente al script build.sh para sistemas Windows
    \item Utiliza comandos nativos de Windows
    \item Genera la misma estructura de archivos
\end{itemize}

\subsubsection{create-standalone-app.sh}
\begin{itemize}
    \item Crea una aplicación independiente para macOS
    \item Utiliza jpackage para generar un archivo .app
    \item Incluye todas las dependencias necesarias
    \item Genera un ejecutable completamente autónomo
\end{itemize}

\subsection{Configuración de compilación}

El proceso de compilación requiere:

\begin{itemize}
    \item Configuración correcta de las variables de entorno Java
    \item Acceso a las librerías JavaFX en \texttt{lib/javafx-sdk-17.0.12/}
    \item Permisos de ejecución en los scripts de construcción
    \item Espacio suficiente en disco para archivos temporales
\end{itemize}

